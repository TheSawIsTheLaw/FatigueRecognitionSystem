\section*{ЗАКЛЮЧЕНИЕ}
\addcontentsline{toc}{section}{ЗАКЛЮЧЕНИЕ}

Была описана проблематика стресса на рабочем месте и метод распознавания усталости пользователя на АРМ с использованием доступных технологий определения усталости , построенных на анализе отдельных характеристик и действий.

Были описаны термины предметной области: усталость, хроническая усталость, стресс. Также рассмотрены стадии общего адаптационного синдрома и понятие профессионального стресса. Проведенная аналитическая работа позволила обозначить проблему: на рабочем месте человек испытывает стресс, избыток которого может навредить человеку и работодателю.

Была формализована цель прототипирования метода СРУ на АРМ: предупреждение наступления стадии истощения посредством регистрации момента наступления стадии тревоги или стадии тревоги и затем устойчивости. Назначение данного метода заключается в выявлении стадии истощения организма для его дальнейшего плодотворного функционирования.

Проведён анализ действий и характеристик, позволяющих определить усталость пользователя АРМ. Для регистрации действий и характеристик были выбраны внешние и носимые устройства.

Рассмотрены следующие внешние устройства, позволяющие определить усталость пользователя АРМ: клавиатура, координатное устройство типа мышь, веб-камера, микрофон.

Рассмотрен метод распознавания усталости пользователя по клавиатурному почерку, который определяется по времени между нажатиями клавиш. 

Рассмотрен метод распознавания усталости пользователя по разнице скорости перемещения курсора и оценки среднего времени перемещения курсора мыши между элементами интерфейса.

Рассмотрен метод распознавания усталости с использованием изображений, получаемых от веб-камеры, который строится на использовании технологий машинного обучения и предварительно обученных на наборах данных моделей.

Рассмотрен метод распознавания усталости с использованием аудио\\-потока, построенного на анализе связной речи человека.

Рассмотрены биофизические факторы, позволяющие определить усталость пользователя: частота пульса, возраст сосудистой системы, индекс стресса (индекс Баевского). В качестве задействованной в системе характеристики был выбран индекс стресса. В качестве устройства для определения выбранной характеристики могут использоваться смарт-часы, поддерживающие подобную функцию.

Таким образом, были определены методы снятия выделенных действий и характеристик.

Формализованный прототип метода СРУ на АРМ включил в себя:
\begin{itemize}[leftmargin=1.6\parindent]
\item хранение и анализ данных, получаемых от клавиатуры;
\item хранение и анализ данных об индексе стресса субъекта. 
\end{itemize}

\pagebreak