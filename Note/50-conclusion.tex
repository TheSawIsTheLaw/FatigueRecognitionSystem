\section*{ЗАКЛЮЧЕНИЕ}
\addcontentsline{toc}{section}{ЗАКЛЮЧЕНИЕ}

Были рассмотрены понятия усталости, хронической усталости, стресса и профессионального стресса, формализована цель разработанного метода.

Был проведен анализ существующих методов определения усталости:

\begin{itemize}[leftmargin=1.6\parindent]
\item анализ клавиатурного почерка;
\item анализ скорости печати и количества ошибок;
\item анализ использования координатного устройства мышь с использованием модели искусственной нейронной сети;
\item анализ внешнего состояния пользователя;
\item анализ речевых характеристик пользователя;
\item анализ виброакустических шумов при наборе текста или использовании мыши.
\end{itemize}

Был проведен анализ действий и характеристик, позволяющих определить усталость пользователя автоматизированного рабочего места. Выделены биофизические факторы, позволяющие определить усталость оператора: частота пульса и возраст сосудистой системы, индекс Баевского. Также рассмотренные существующие методы позволили в качестве рассматриваемых действий пользователя выбрать нажатия на клавиши клавиатуры и мыши. Было определено, что система будет проводить кластеризацию методом c-средних по значениям скорости печати, скорости передвижения курсора между двумя нажатиями, а также по значениям времени реакции пользователя.

Были определены методы снятия выделенных действий и характеристик. Каждое нажатие на клавишу клавиатуры характеризуется наименованием нажатой клавиши и временной меткой. Каждое нажатие на клавишу мыши характеризуется координатами экрана, в которых действие было совершено, номером нажатой клавиши и временной меткой. Тест на реакцию характеризуется определенным временем реакции и временной меткой, когда значение было получено.

Был разработан метод распознавания усталости оператора, который включил в себя хранение и анализ данных, получаемых от клавиатуры и мыши. Было определено, что формирование нечетких кластеров происходит на этапе синхронизации поступающих данных о действиях пользователя с данными о скорости его реакции в отдельные моменты времени. Отмечено, что в дальнейшем полученная модель используется до ее актуализации, которая происходит в случае появления ложных срабатываний или радикального изменения поведения объекта.

Разработанный метод был реализован. Реализация включила в себя три модуля: логирования действий оператора, анализа данных и серверного приложения. Были приведены особенности реализации каждого модуля, диаграммы классов.

Результаты исследования позволили определить, что выбор в пользу InfluxDB имеет и экспериментальные подтверждения. Поставленный эксперимент показал преимущство скорости исполнения запросов в базу данных InfluxDB над скоростью исполнения запросов в базу данных Postgres с использованием языка программирования данных. В среднем InfluxDB позволил получить ответ в $\approx 4.53$ раза быстрее.

Также было проведено исследование в области сравнения количества успешных определения работоспособности пользователя путем варьирования фактора нечеткости. В результате было определено, что при определении усталости с использованием клавиатуры на заданной выборке при варьировании фактора нечеткости было получено $\approx 66\%$ верных результатов, что на $16\%$ больше, чем при использовании мыши. Эффективным устройством для распознавания усталости в данных испытаниях была признана клавиатура.

\pagebreak