%----------------------- Преамбула -----------------------
\usepackage{extsizes} % Для добавления в параметры класса документа 14pt

\usepackage{extsizes} % Для добавления в параметры класса документа 14pt

% Для работы с несколькими языками и шрифтом Times New Roman по-умолчанию
\usepackage[english,russian]{babel}
\usepackage{fontspec}
\setmainfont{Times New Roman}

% ГОСТовские настройки для полей и абзацев
\usepackage[left=30mm,right=10mm,top=20mm,bottom=20mm]{geometry}
\usepackage{misccorr}
\usepackage{indentfirst}
\usepackage{enumitem}
\setlength{\parindent}{1.25cm}
%\setlength{\parskip}{1em} % поменять
%\linespread{1.3}
\renewcommand{\baselinestretch}{1.5}
\setlist{nolistsep} % Отсутствие отступов между элементами \enumerate и \itemize

% hz
\usepackage{multirow}

% Дополнительное окружения для подписей
\usepackage{array}
\newenvironment{signstabular}[1][1]{
	\renewcommand*{\arraystretch}{#1}
	\tabular
}{
	\endtabular
}

% Переопределение стандартных \section, \subsection, \subsubsection по ГОСТу;
% Переопределение их отступов до и после для 1.5 интервала во всем документе
\usepackage{titlesec}

\titleformat{\section}[block]
{\bfseries\normalsize\filcenter}{\thesection}{1em}{}

\titleformat{\subsection}[hang]
{\bfseries\normalsize}{\thesubsection}{1em}{}
\titlespacing\subsection{\parindent}{\parskip}{\parskip}

\titleformat{\subsubsection}[hang]
{\bfseries\normalsize}{\thesubsubsection}{1em}{}
\titlespacing\subsubsection{\parindent}{\parskip}{\parskip}

\usepackage{float}
\usepackage{setspace}
\onehalfspacing % Полуторный интервал

% Работа с изображениями и таблицами; переопределение названий по ГОСТу
\usepackage{caption}
\usepackage{titlesec}
\titleformat{\chapter}{\LARGE\bfseries}{\thechapter}{20pt}{\LARGE\bfseries}
\titleformat{\section}{\centering\Large\bfseries}{\thesection.\quad}{20pt}{\Large\bfseries}
\usepackage{listings}
\usepackage[utf8x]{inputenc}  
% Выравнивание "где" по ГОСТу для формул
\usepackage{eqexpl}

\usepackage{graphicx}
\usepackage{slashbox} % Диагональное разделение первой ячейки в таблицах

% Цвета для гиперссылок и листингов
\usepackage{color}

% Для нормальной нумерации картинок, разделённой по section'ам
\usepackage{chngcntr}
\counterwithin{figure}{section}

% Гиперссылки \toc с кликабельностью
\usepackage{hyperref}

\hypersetup{
	linktoc=all,
	linkcolor=black,
	colorlinks=false,
}

% Листинги
\setsansfont{Arial}
\setmonofont{Courier New}

\usepackage{color} % Цвета для гиперссылок и листингов
%\definecolor{comment}{rgb}{0,0.5,0}
%\definecolor{plain}{rgb}{0.2,0.2,0.2}
%\definecolor{string}{rgb}{0.91,0.45,0.32}
%\hypersetup{citecolor=blue}
\hypersetup{citecolor=black}

\lstdefinelanguage{kotlin}{
  comment=[l]{//},
  commentstyle={\color{black}\ttfamily},
  emph={filter, first, firstOrNull, forEach, lazy, map, mapNotNull, println},
  emphstyle={\color{black}},
  identifierstyle=\color{black},
  keywords={!in, !is, abstract, actual, annotation, as, as?, break, by, catch, class, companion, const, constructor, continue, crossinline, data, delegate, do, dynamic, else, enum, expect, external, false, field, file, final, finally, for, fun, get, if, import, in, infix, init, inline, inner, interface, internal, is, lateinit, noinline, null, object, open, operator, out, override, package, param, private, property, protected, public, receiveris, reified, return, return@, sealed, set, setparam, super, suspend, tailrec, this, throw, true, try, typealias, typeof, val, var, vararg, when, where, while},
  keywordstyle={\color{black}\bfseries},
  morecomment=[s]{/*}{*/},
  morestring=[b]",
  morestring=[s]{"""*}{*"""},
  ndkeywords={@Deprecated, @JvmField, @JvmName, @JvmOverloads, @JvmStatic, @JvmSynthetic, Array, Byte, Double, Float, Int, Integer, Iterable, Long, Runnable, Short, String, Any, Unit, Nothing},
  ndkeywordstyle={\color{black}\bfseries},
  sensitive=true,
  stringstyle={\color{black}\ttfamily},
}

\usepackage{listings}
\usepackage{xcolor}

\lstset{
	basicstyle=\scriptsize\ttfamily\small,
	breaklines=true,
	commentstyle=\color{black},
	frame=single,
	linewidth=\textwidth,
	keywordstyle=\color{black}\bfseries,
	numbers=left,
	numbersep=5pt,
	numberstyle=\small\ttfamily\color{black},
	showstringspaces=false,
	stringstyle=\color{black},
	tabsize=4
}

% plot
\usepackage{pgfplots}
\usepackage{filecontents}
\usetikzlibrary{datavisualization}
\usetikzlibrary{datavisualization.formats.functions}

\begin{filecontents}{postgres.dat}
10000 188
12000 248
14000 274
16000 261
18000 266
20000 284
22000 316
24000 340
26000 363
28000 410
\end{filecontents}

\begin{filecontents}{influx.dat}
10000 64
12000 65
14000 60
16000 60
18000 61
20000 68
22000 76
24000 65
26000 65
28000 67
\end{filecontents}

\newcommand{\code}[1]{\texttt{#1}}

\usepackage{ulem} % Нормальное нижнее подчеркивание
\usepackage{hhline} % Двойная горизонтальная линия в таблицах
\usepackage[figure,table]{totalcount} % Подсчет изображений, таблиц
\usepackage{rotating} % Поворот изображения вместе с названием
\usepackage{lastpage} % Для подсчета числа страниц

\makeatletter
\renewcommand\@biblabel[1]{#1.}
\makeatother

\captionsetup[figure]{name={Рисунок},labelsep=endash}
\captionsetup[table]{singlelinecheck=false, labelsep=endash}

\usepackage{color}
\usepackage[cache=false, newfloat]{minted}

\usepackage{amsmath}
\usepackage{slashbox}
