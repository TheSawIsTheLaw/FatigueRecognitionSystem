\section{Аналитическая часть}

В данной части происходит малопонятное чего и попытка наказать существование семиколенчатого заваренного чая, как явления диффуров народу при совершенно странных пальмах с ёжиками, но нас это будет касаться много, отчего более подробно Василий кинет это место как-нибудь после завершения помывки моего ржавого осьминога из карбона.

\subsection{Устройства взаимодействия пользователя АРМ с системой}
Внешние устройства (периферия) - совокупность дополнительных устройств персонального компьютера (ПК), расширяющих его функционал \cite{2hpc}.

Среди внешних устройств, которые относятся к устройствам ввода, то есть органам управления ПК, относят\cite{2hpc}:
\begin{itemize}
\item клавиатуру,
\item мышь,
\item графический планшет,
\item игровые манипуляторы,
\item веб-камеру,
\item микрофон.
\end{itemize}

В рамках данной работы превентивно исключаются игровые манипуляторы и графические планшеты ввиду их нераспространённости на рабочих местах большинства пользователей.

Таким образом, потребуется провести анализ того, какие характеристики и иная информация может быть получена от клавиатуры, координатного устройства, веб-камеры и микрофона с целью определения усталости пользователя.

\subsubsection{Клавиатура}
Клавиатура - одно из наиболее используемых внешних устройств для взаимодействия пользователя с АРМ. Характеристики использования могут включать в себя:
\begin{itemize}
\item скорость печати,
\item частоту исправления ошибок в уже напечатанных словах.
\end{itemize}

\subsubsection{Координатное устройство для управления курсором}
Координатное устройство или компьютерная мышь используется для взаимодействия с оконным интерфейсом операционной системы и программ.
Характеристики использования данного устройства могут включать в себя:
\begin{itemize}
\item частоту нажатия клавиш координатного устройства,
\item скорость движения.
\end{itemize}

\subsubsection{Веб-камера}
В случае, если пользователь имеет возможность находиться непосредственно перед камерой, установленной в или рядом с персональным компьютером, данное устройство способно регистрировать:

\begin{itemize}
\item мимику,
\item жесты,
\item частоту моргания,
\item ровность и размеренность дыхания,
\item моменты, когда глаза пользователя закрыты.
\end{itemize}

\subsubsection{Микрофон}
Микрофон позволяет регистрировать аудиопоток, исходящий от пользователя и его окружения. Данное устройство может регистрировать:
\begin{itemize}
\item тембр голоса,
\item скорость речи.
\end{itemize}

Целесообразность использования данного устройства в качестве регистрирующего сомнительна. Связано это с большим количеством посторонних звуков и шума в офисных помещениях, а также в иных местах, где система предполагается к размещению. По данной причине указанное внешнее устройство далее рассматриваться не будет.

\subsubsection*{Вывод}
Цель данной работы - создание благоприятных условий на рабочих местах. Решение задачи установления зависимости работоспособности и физиологического состояния работника может позволить сохранить здоровье и работоспособность трудящихся, решить проблему повышения эффективности работы, заболеваемости на высоконагруженных трудовых местах и иных вопросов здравоохранения.

Итоговый список характеристик, которые можно получить непосредственно на рабочем месте, предоставлен в таблице \ref{table: charTable}.

\begin{table}[H]
	\begin{center}
	\captionof{table}{\label{table: charTable} Характеристики для снятия}
		\begin{tabular}{|p{5cm} | p{4cm} | p{5cm} |} 
 			\hline
			Характеристика & Средство снятия & Периодичность снятия\\ [0.5ex] 
 			\hline\hline
 			Уровень адреналина в крови & Клинический\newline анализ крови & Каждый период возникновения учащённого дыхания \\
 			\hline
 			Уровень кортизола & Анализ слюны\newline человека & В периоды фиксации стресса каждые 5-10 минут \\
 			\hline
 			Уровень дегидроэпиандростерона & Анализ слюны человека & В периоды фиксации стресса каждые 5-10 минут \\
 			\hline
 			Пульс & Пульсометр, смарт-часы с пульсометром & Каждую 1 минуту \\
 			\hline
 			Артериальное\newline давление & Тонометр,\newline смарт-часы с\newline тонометром & Каждые 5 минут, при отклонениях - каждую 1 минуту \\
 			\hline
 			Скорость печати & Клавиатура & В периоды активности пользователя \\
 			\hline
 			Частота исправления ошибок в напечатанных словах & Клавиатура & В периоды активности пользователя \\
 			\hline
 			Частота нажатий клавиш координатного устройства & Компьютерная мышь & В периоды активности пользователя \\
 			\hline
 			Скорость движения координатного устройства & Компьютерная мышь & В периоды активности пользователя \\
 			\hline
 			Мимика & Веб-камера & Непрерывно\\
 			\hline
 			Жесты & Веб-камера & Непрерывно\\
 			\hline
 			Частота моргания & Веб-камера & Непрерывно \\
 			\hline
 			Ровность и размеренность дыхания & Смарт-часы,\newline веб-камера & Непрерывно \\
 			\hline
 			Тембр голоса & Микрофон & В периоды активности пользователя \\
 			\hline
 			Скорость речи & Микрофон & В периоды активности пользователя \\
 			\hline
			\end{tabular}
	\end{center}
\end{table}

\subsection{Подраздел}

Список:

\begin{itemize}[leftmargin=1.6\parindent]
	\item[---] первое;
	\item[---] второе;
	\item[---] пятое;
	\item[---] десятое.
\end{itemize}

Формула:

\begin{equation}
c^2 = a^2 + b^2
\end{equation}

Ссылаемся на рисунок \ref{fig:a1}. Информация из источника \cite{MSD}.

\begin{figure}[hbtp]
	\centering
	\includegraphics[width=\textwidth]{img/golang.png}
	\caption{Пример рисунка}
	\label{fig:a1}
\end{figure}

\begin{code}
	\captionof{listing}{Пример кода}
	\label{code:1}
	\inputminted
	[
	frame=single,
	framerule=0.5pt,
	framesep=20pt,
	fontsize=\small,
	tabsize=4,
	linenos,
	numbersep=5pt,
	xleftmargin=10pt,
	]
	{text}
	{code/main.go}
\end{code}

\pagebreak