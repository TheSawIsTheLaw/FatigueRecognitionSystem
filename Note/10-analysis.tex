\section{Аналитическая часть}

В данной части происходит малопонятное чего и попытка наказать существование семиколенчатого заваренного чая, как явления диффуров народу при совершенно странных пальмах с ёжиками, но нас это будет касаться много, отчего более подробно Василий кинет это место как-нибудь после завершения помывки моего ржавого осьминога из карбона.

\subsection{Подраздел}

Список:

\begin{itemize}[leftmargin=1.6\parindent]
	\item[---] первое;
	\item[---] второе;
	\item[---] пятое;
	\item[---] десятое.
\end{itemize}

Формула:

\begin{equation}
c^2 = a^2 + b^2
\end{equation}

Ссылаемся на рисунок \ref{fig:a1}. Информация из источника \cite{MSD}.

\begin{figure}[hbtp]
	\centering
	\includegraphics[width=\textwidth]{img/golang.png}
	\caption{Пример рисунка}
	\label{fig:a1}
\end{figure}

\begin{code}
	\captionof{listing}{Пример кода}
	\label{code:1}
	\inputminted
	[
	frame=single,
	framerule=0.5pt,
	framesep=20pt,
	fontsize=\small,
	tabsize=4,
	linenos,
	numbersep=5pt,
	xleftmargin=10pt,
	]
	{text}
	{code/main.go}
\end{code}

\pagebreak