\section{Аналитическая часть}

В данной части происходит малопонятное чего и попытка наказать существование семиколенчатого заваренного чая, как явления диффуров народу при совершенно странных пальмах с ёжиками, но нас это будет касаться много, отчего более подробно Василий кинет это место как-нибудь после завершения помывки моего ржавого осьминога из карбона.

\subsection{Устройства взаимодействия пользователя АРМ с системой}
Внешние устройства (периферийные устройства) -- устройства ввода-вывода, распечатки, хранения и передачи информации, связанные функционально с центральным процессором в соответствии со структура ЭВМ (или системы ЭВМ). \cite{encDic}

К внешним устройствам, которые являются устройствами ввода, то есть органами управления персональным компьютером, относят:
\begin{itemize}[leftmargin=1.6\parindent]
\item[1)] Клавиатуру;
\item[2)] Мышь;
\item[3)] Графический планшет;
\item[4)] Веб-камеру;
\item[5)] Микрофон;
\item[6)] Игровой манипулятор.
\end{itemize}

Из указанных выше устройств в рассмотрение не войдут:
\begin{itemize}[leftmargin=1.6\parindent]
\item[1)] Графический планшет;
\item[2)] Игровой манипулятор.
\end{itemize}

Данное решение связанно с тем, что использование подобного рода устройств указывает на особый род работы. В требованиях к разрабатываемой системе указывается возможность её использования для операторов любых автоматизированных рабочих мест, поэтому использование в решении поставленной задачи информации, получаемой с графических планшетов и игровых манипуляторов, не является целесообразным.

Согласно проведённым исследованиям \cite{recognitionOfPsycho} признаки голоса, клавиатурного почерка и характера работы исследуемого или контролируемого субъекта с компьютерной мышью содержат информацию о психофизиологических состояниях оператора: нормальное, усталость, опьянение, возбужденное, расслабленное (сонное).

\subsubsection{Клавиатура}
Клавиатура -- одно из наиболее часто используемых внешних устройств для взаимодействия пользователя с персональным компьютером.

Клавиатурный почерк -- это подвид поведенческой подгруппы аутентификации по неотчуждаемым признакам. \cite{keystroke}

Клавиатурный почерк определяется по времени между нажатиями клавиш. При снятии биометрического шаблона клавиатурного почерка измеряют время нажатия двух, трех или четырех последовательных клавиш, сохраняют его и на основе полученных значений строят математические модели для сравнения шаблонов нескольких пользователей. \cite{intrusionDetection} Для системы входными данными будут два биометрических шаблона -- эталонного и кандидата. Результат работы системы -- рейтинг доверия к биометрическому шаблону кандидата, который является критерием схожести двух переданных шаблонов.

Различают два вида распознавания клавиатурного почерка: распознавание на статическом тексте (пароль или известная кодовая фраза), распознавание при вводе псевдослучайного текста. \cite{keystroke}

Математическая задача распознавания на фиксированном тексте может быть формализована. Для её выполнения потребуется \cite{keystroke}:
\begin{itemize}
\item[1)] Собрать информацию о времени между нажатиями соседних клавиш в тексте;
\item[2)] Сформировать из полученных данных вектор фиксированной размерности;
\item[3)] По кластерной модели или другой модели сравнения двух векторов сравнить сформированный вектор и вектор-эталон для этого же текста от этого же пользователя.
\end{itemize}

Для задачи распознавания клавиатурного почерка при вводе псевдослучайного текста не существует надежных моделей формирования пользовательского шаблона и вычисления рейтинга. Время работы подобных систем не позволяет в реальном времени оценить ситуацию и выдает результат через десятки минут, а память, занимаемая векторами, лишь продолжает расти. \cite{keystroke}

Выделено два основных численных показателя, которые определяют качество биометрической системы \cite{keystroke}:
\begin{itemize}[leftmargin=1.6\parindent]
\item[1)] Ошибка первого рода (FRR, количество ложноотрицательных);
\item[2)] Ошибка второго рода (FAR, количество ложноположительных).
\end{itemize}

Ошибка первого рода -- это вероятность ложного отказа в доступе. Данная ошибка имеет место при возникновении следующих факторов \cite{keystroke}: 
\begin{itemize}[leftmargin=1.6\parindent]
\item[1)] Повреждение рук пользователя;
\item[2)] Ненормальное психофизического состояния человека (усталость, алкогольное или наркотическое опьянение, приступ гнева).
\end{itemize}

Ошибка второго рода -- это вероятность ложного допуска. Данная ошибка имеет место при возникновении следующих факторов \cite{keystroke}:
\begin{itemize}[leftmargin=1.6\parindent]
\item[1)] Заданные возможные отклонения от допустимых значений при распознавании пользователя были заданы неверно;
\item[2)] Нарушитель сумел скопировать метрики поведения пользователя и обойти систему контроля.
\end{itemize}

В описанных ошибках для проектируемой системы важнейшую роль играет ошибка первого рода. Это связано с тем, что данная ошибка способна позволить распознать состояние усталости оператора в случае единоличного пользования системой.

\subsubsection{Координатное устройство для управления курсором}
Координатное устройство или компьютерная мышь используется для взаимодействия с оконным интерфейсом операционной системы и программ.

Особенности работы с мышью можно оценить, анализируя траектории передвижения курсора мыши по экрану между элементами интерфейса и среднее время выполнения передвижения. \cite{recognitionOfPsycho}

Оценка среднего времени перемещения курсора мыши между элементами интерфейса выполняется с использованием адаптированной для данной задачи формулы \ref{eq:fitts} закона Фиттса \cite{fitts}. Данный закон связывает время движения к наблюдаемой цели с точностью движения и с расстоянием до наблюдаемой цели. Чем дальше или точнее выполняется движение руки субъекта, тем больше коррекции необходимо для его выполнения, и соответственно, больше времени требуется субъекту для внесения этой коррекции. Фактическое время перемещения не должно совпадать с оценкой, вычисляемой по формуле \ref{eq:fitts}, а должно отличаться на величину $\Delta T$, которую используют как один из идентификцирующих признаков. \cite{mouseMethod}

\begin{equation}
\label{eq:fitts}
T = b \cdot \log_{2}(\frac{D}{W} + 1)
\end{equation}

В формуле \ref{eq:fitts} $b$ - величина, зависящая от типичной скорости движения курсора мыши (отношение средней скорости движения мыши по экрану, осуществляемого субъектом, к установленному в операционной системе коэффициенту чувствительности мыши), $D$ - дистанция перемещения курсора между элементами интерфейса ( в пикселях), а $W$ - ширина элемента интерфейса, к которому направляется курсор (в пикселях).

Также в качестве признаков предлагается использовать амплитуды первых десяти низкочастотных гармоник функции скорости перемещения курсора мыши по экрану $V_{xy}(t)$, вычисляемой по формуле \ref{eq:harmonic} разложение функции производится с помощью быстрого преобразования Фурье, тем самым достигается нормирование чувастков пути курсора по времени. Каждый участок приводится к длительности в 0,5 секунд. Амплитуды нормируются по энергии функции $V_{xy}(t)$, вычисляемой в соответствии с формулой \ref{eq:normir}, данная операция осуществляется для того, чтобы привести все траектории перемещений курсора между элементами интерфейса к единому масштабу. Аналогичные операции осуществляются по отношению к функциям координат курсора $x(t)$ и $y(t)$, однако прещдварительно данные функции переводятся в другую систему координат, где началом координат является центр элемента интерфейса, на который было произведено нажатие, ось абсцисс располагается в направлении центра элемента интерфейса, по отношению к которому произхводится перемещение курсора. Это необходимо выполнять, чтобы избавиться от наклона линий, связывающих элементы интерфейса, относительно исходной координатной плоскости (то есть зависимости координат от угла наклона). \cite{mouseMethod}

\begin{equation}
\label{eq:harmonic}
V_{xy}(t) = \sqrt{((x(t_{i+1}) - x(t_i))^2 + (y(t_{i+1})-y(t_i))^2)^2}
\end{equation}

В формуле \ref{eq:harmonic} $x$ и $y$ - координаты курсора, $t_i$ - $i$-ый момент времени регистрации координат курсора (регистрация координат курсора зависит от производительности компьютера).

\begin{equation}
\label{eq:normir}
E_s = \int\limits_{\infty}^{-\infty} A^2(\omega)dt
\end{equation}

В формуле \ref{eq:normir} $A(\omega)$ -- амплитуды гармоники с частотой $\omega$ функции $V_{xy}(t)$.

\subsubsection{Веб-камера}
Основные исследования в области использования видео-изображений для определения опасных состояний усталости проводятся для реализации систем распознавания усталости водителя.

Признаки состояний ослабленного внимания и усталости у водителя характеризуется следующими наблюдаемыми параметрами \cite{videoMethod}:
\begin{itemize}[leftmargin=1.6\parindent]
\item[1)] Поворот головы влево/вправо по отношению к туловищу;
\item[2)] Наклон головы вперед относительно туловища;
\item[3)] Продолжительность моргания век;
\item[4)] Частота моргания век;
\item[5)] Степень открытости рта человека (признаки зевоты).
\end{itemize}

\subsubsection{Микрофон}
Микрофон позволяет регистрировать аудиопоток, исходящий от пользователя и его окружения. Данное устройство может регистрировать:
\begin{itemize}
\item тембр голоса,
\item скорость речи.
\end{itemize}

Целесообразность использования данного устройства в качестве регистрирующего сомнительна. Связано это с большим количеством посторонних звуков и шума в офисных помещениях, а также в иных местах, где система предполагается к размещению. По данной причине указанное внешнее устройство далее рассматриваться не будет.

\subsubsection*{Вывод}
Цель данной работы -- создание благоприятных условий на рабочих местах. Решение задачи установления зависимости работоспособности и физиологического состояния работника может позволить сохранить здоровье и работоспособность трудящихся, решить проблему повышения эффективности работы, заболеваемости на высоконагруженных трудовых местах и иных вопросов здравоохранения.

Итоговый список характеристик, которые можно получить непосредственно на рабочем месте, предоставлен в таблице \ref{table: charTable}.

\begin{table}[H]
	\begin{center}
	\centering
	\captionof{table}{\label{table: charTable} Характеристики для снятия}
		\begin{tabular}{|p{5cm} | p{4cm} | p{5cm} |} 
 			\hline
			Характеристика & Средство снятия & Периодичность снятия\\ [0.5ex] 
 			\hline\hline
 			Уровень адреналина в крови & Клинический\newline анализ крови & Каждый период возникновения учащённого дыхания \\
 			\hline
 			Уровень кортизола & Анализ слюны\newline человека & В периоды фиксации стресса каждые 5-10 минут \\
 			\hline
 			Уровень дегидроэпиандростерона & Анализ слюны человека & В периоды фиксации стресса каждые 5-10 минут \\
 			\hline
 			Пульс & Пульсометр, смарт-часы с пульсометром & Каждую 1 минуту \\
 			\hline
 			Артериальное\newline давление & Тонометр,\newline смарт-часы с\newline тонометром & Каждые 5 минут, при отклонениях -- каждую 1 минуту \\
 			\hline
 			Скорость печати & Клавиатура & В периоды активности пользователя \\
 			\hline
 			Частота исправления ошибок в напечатанных словах & Клавиатура & В периоды активности пользователя \\
 			\hline
 			Частота нажатий клавиш координатного устройства & Компьютерная мышь & В периоды активности пользователя \\
 			\hline
 			Скорость движения координатного устройства & Компьютерная мышь & В периоды активности пользователя \\
 			\hline
 			Мимика & Веб-камера & Непрерывно\\
 			\hline
 			Жесты & Веб-камера & Непрерывно\\
 			\hline
 			Частота моргания & Веб-камера & Непрерывно \\
 			\hline
 			Ровность и размеренность дыхания & Смарт-часы,\newline веб-камера & Непрерывно \\
 			\hline
 			Тембр голоса & Микрофон & В периоды активности пользователя \\
 			\hline
 			Скорость речи & Микрофон & В периоды активности пользователя \\
 			\hline
			\end{tabular}
	\end{center}
\end{table}
\pagebreak