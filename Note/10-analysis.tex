\section{Аналитический раздел}
В данном разделе...

\subsection{Термины предметной области}
\subsubsection{Усталость}
\textit{Усталость} --- ощущение физической усталости и отсутствия энергии, которое нарушает повседневную физическую и социальную жизнь, не связанное с умственным переутомлением, депрессией, сонливостью, нарушением двигательных функций. Выделяют два феномена: физическую усталость, которая заключается в снижении способности поддержания физической активности, и психическую, которая заключается в уменьшении способности выполнения умственных задач. \cite{fatigueAsSymptom}

Диагностирование усталости у пациента затруднительна в силу отсутствия общепринятой симптоматики и определения синдрома. На момент 2021 года объективных критериев усталости не выделено. Также отсутствуют инструментальные методы оценки усталости. \cite{fatigueAsSymptom}

\subsubsection{Хроническая усталость}
\textit{Синдром хронической усталости (СХУ)} --- это инвалидизирующее клиническое состояние, которое характеризуется стойкой усталостью после физических нагрузок, и сопровождающееся симптомами, связанными с когнитивной, иммунологической, эндокринологической и автономной дисфункцией. \cite{syndromOfChrono}

Согласно исследованиям Института медицины США, от 836 000 до 2.5 миллионов американцев страдают СХУ. Данный факт приводит к финансовым затратам от 17 до 24 миллиардов долларов в год, а каждая семья --- 20 000 долларов. Уровень безработицы среди болеющих составляет от 35\% до 69\%. \cite{fatigueChronoInvestigation}

Наличие синдрома у пациента определяется наличием необъяснимой стойкой хронической усталости. К симптомам заболевания относят \cite{syndromOfChrono}:
\begin{itemize}[leftmargin=1.6\parindent]
\item нарушение памяти или концентрации внимания;
\item фарингит;
\item болезненные при пальпации шейные или подмышечные лимфоузлы;
\item болезненность или скованность мышц;
\item болезненность суставов;
\item вновь возникающая головная боль или изменение ее характеристик;
\item сон, не приносящий ощущения восстановления;
\item усугубление усталости, продолжающееся более 24 часов.
\end{itemize}

Более трех одновременно диагностируемых симптомов, сохраняющихся в течение более 5 месяцев, определяют наличие СХУ. \cite{syndromOfChrono}


\subsubsection{Стресс}
\textit{Стресс} --- это реакция человеческого организма на неблагоприятные воздействия внешней среды, которая носит психофизиологический характер. К подобным неблагоприятным воздействиям можно определить: конфликтные ситуации, потеря близких людей, насилие, несправедливость, ограничение потребностей. \cite{neuroPhysicalMechasmsOfStress}

Стрессовая реакция заключается в активации парасимпатического и симпатического отдела вегетативной нервной системы, в следствие чего стимулируется одна из трех психофизиологических осей стресса. \cite{neuroPhysicalMechasmsOfStress}

Активация симпатической нервной системы заключается в общем возбуждении внутреннего органа, осуществляющего реагирование на стресс, а активация парасимпатического характера проявляется в виде чрезмерного возбуждения внутреннего органа, сменяющегося на торможение, замедление или нормализацию. Проявление симпатической реакции, в случае разрешения стрессовой ситуации, может смениться парасимпатической заторможенностью, возвращающей органы к нормальному функционированию. \cite{neuroPhysicalMechasmsOfStress}

При продолжении течения стрессовой реакции задействуется нейроэндокринный процесс, который активирует все ресурсы организма: выделение в кровь адреналина и норадреналина, выброс гормонов. Данная реакция приводит к большему возбуждению, увеличению сердечного выброса, подъему артериального давления, снижению кровотока к почкам, повышению уровня жирных кислот в плазме и увеличению уровня холестерина. \cite{neuroPhysicalMechasmsOfStress}

\subsubsection{Стадии общего адаптационного синдрома}

\textit{Общий адаптационный синдром} --- это сочетание стереотипных реакций, возникающих в организме в ответ на действие стрессоров и обеспечивающих ему устойчивость не только к стрессорному агенту, но и по отношению к другим болезнетворным факторам. \cite{stressAndPatology}

Общий адаптационный синдром протекает в три стадии: тревоги, устойчивости и истощения.

Стадия тревоги представляет собой мобилизацию ресурсов организма. Она имеет продолжительность от 6 до 48 часов и включает в себя стадии шока и противошока. Стадия противошока характеризуется мобилизацией основных функциональных систем организма: нервной, симпато-адреналовой, эндокринной и адрено-кортитропной. \cite{stressAndPatology}

При длительном воздействии стрессора на организм наступает стадия устойчивости (адаптации). На данной стадии организм становится более устойчивым к действию раздражителя и другим патогенным факторам, повышается образование и секреция глюкокортикоидов. Поддерживается состояние гомеостаза в присутствии стрессора. \cite{stressAndPatology}

При длительном воздействии стрессора адаптивные механизмы, участвующие в поддержании резистентности, исчерпывают себя, и наступает стадия истощения организма. Данная стадия не является обязательной и стрессовая ситуация может быть окончена до ее наступления. Протекание реакции заключается в активизации эндокринной системы и обеднении коры надпочечников. \cite{stressAndPatology}

В стрессовой ситуации важно выявить наступление стадии истощения организма для его дальнейшего плодотворного функционирования. Подобные действия способны позволить корректировать нагрузку на человека, а также эффективно управлять его активностью.

\subsubsection{Профессиональный стресс}
Профессиональный стресс возникает при воздействии на работника эмоционально-отрицательных или экстремальных факторов окружающей среды при выполнении профессиональной деятельности. Подобные стрессы влияют на здоровье человека и производительность труда, в особенности на сотрудников организаций, которые работают в условиях систематических рабочих и информационных перегрузок. \cite{professionalStress}

Среди причин возникновения профессионального стресса различают следующие организационные факторы в трудовой деятельности человека \cite{professionalStress}:
\begin{itemize}[leftmargin=1.6\parindent]
\item режим трудовой деятельности (большая рабочая нагрузка, функциональная перегрузка, сроки выполнения);
\item нерациональная организация труда и рабочего места (нечеткие ограничения полномочий и обязанностей, неоднозначные требования к выполняемой работе);
\item неудовлетворительные условия труда персонала (монотонная работа, отсутствие информационных и материальных ресурсов);
\item отсутствие эффективной системы мотивации и стимулирования (недостаточное вознаграждение за труд, риск штрафных санкций противоречие интересов работников и их функциональных обязанностей);
\item неэффективный стиль управления, 
\item отсутствие или недостаток поддержки со стороны начальства;
\item неудовлетворительная социально-психологическая атмосфера (неблагоприятный социально-психологический климат в коллективе, отсутствие или недостаток поддержки со стороны коллег, нарушение внутригрупповых норм поведения);
\item лишение перспектив карьерного роста, бесперспективность работы.
\end{itemize}

% Таким образом, понятие профессионального стресса может быть расширенно как психическое напряжение, связанное с преодолением несовершенства организационных условий труда, с высокими нагрузками при выполнении профессиональных обязанностей на рабочем месте в конкретной организационной структуре, а также с поиском новых неординарных рабочих решений при форс-мажорных обстоятельствах. \cite{professionalStress}

К последствиям профессионального стресса относят \cite{professionalStress}:
\begin{itemize}[leftmargin=1.6\parindent]
\item низкую эффективность работы;
\item апатичность;
\item консерватизм;
\item пессимистичность;
\item избегание коммуникации на работе;
\item физическое истощение, болезненность.
\end{itemize}

Профилактика профессиональных стрессов может включать в себя \cite{professionalStress}:
\begin{itemize}[leftmargin=1.6\parindent]
\item рациональное распределение рабочей нагрузки;
\item четкое описание должностных обязанностей, полномочий;
\item создание комфортной социально-психологической среды в трудовом коллективе и организации, устранение конфликтов;
\item предоставление возможности карьерного роста сотрудников в рамках организации;
\item предоставление информации о системе оценки труда и системе мотивации трудовой деятельности, поощрение лучших сотрудников.
\end{itemize}

Последствия профессиональных стрессов носят как физиологический, так и психологический характер. Предупреждение наступления стадии истощения с использованием информации, поступающей от пользователя, позволит сохранить здоровье и работоспособность трудящихся, а также решить проблему повышения эффективности работы и заболеваемости на высоконагруженных трудовых местах.

\subsection{Формализация цели метода систематического распознавания усталости на автоматизированном рабочем месте}
В стрессовой ситуации важно выявить наступление стадии истощения организма для его дальнейшего плодотворного функционирования. Подобные действия способны позволить корректировать нагрузку на человека, а также эффективно управлять его активностью.

Для предупреждения наступления стадии истощения требуется определить момент наступления стадии тревоги или стадии тревоги и затем устойчивости. Данные наблюдения позволят определить, продолжается ли воздействие стрессора на организм. В случае возвращения характеристик пользователя к стандартным, построенным на наблюдении вне периода стресса, дальнейших действий не потребуется. Предупреждение о наступлении стадии истощения предполагается в случае, когда характеристики пользователя, находящегося в установленной стадии устойчивости, устремляются к нестандартным значениям, что говорит об активизации эндокринной системы, угнетающей функции внутренних органов.

Согласно исследованиям, проведенными Федеральной службой государственной статистики \cite{rosstatInvestigation}, 71.2\% от общей численности населения в городской местности используют персональные компьютеры дома, в то время как 33.9\% --- на рабочих местах.

Подобная статистика предполагает возможность использования в качестве данных о состоянии пользователя для предупреждения состояния усталости характеристик, получаемых от внешних устройств, как на рабочих местах, так и в домашних условиях.

\subsection{Методы определения усталости оператора с использованием устройств автоматизированного рабочего места}
\textit{Внешние устройства (периферийные устройства)} --- устройства, предназначенные для внешней машинной обработки информации (в отличие от преобразований информации, осуществляемых центральным процессором) \cite{polytechDic}. По роду выполняемых операций данный устройства подразделяются на группы \cite{polytechDic}: 
\begin{itemize}[leftmargin=1.6\parindent]
\item устройства подготовки данных (занесения информации на промежуточные носители данных);
\item устройства ввода (считывание информации и её преобразование в кодовую последовательность электрических сигналов, подлежащих передаче в центральный процессор;
\item устройства вывода (регистрация результатов обработки информации или их отображения);
\item устройства хранения больших объемов информации;
\item устройства передачи информации на большие расстояния.
\end{itemize}

К внешним устройствам, которые являются устройствами ввода, то есть органами управления персональным компьютером, относят:
\begin{itemize}[leftmargin=1.6\parindent]
\item клавиатуру;
\item мышь;
\item графический планшет;
\item веб-камеру;
\item микрофон;
\item игровой манипулятор.
\end{itemize}

Из указанных выше устройств в рассмотрение не войдут:
\begin{itemize}[leftmargin=1.6\parindent]
\item графический планшет;
\item игровой манипулятор.
\end{itemize}

Данное решение связанно с тем, что использование подобного рода ус-\\тройств указывает на особый род работы. В требованиях к разрабатываемой системе указывается возможность ее использования для операторов автоматизированных рабочих мест в наиболее распространенных  конфигурациях (например, для офиса), поэтому использование в решении поставленной задачи информации, получаемой с графических планшетов и игровых манипуляторов, не является целесообразным.

Согласно проведенным исследованиям \cite{recognitionOfPsycho} признаки голоса, клавиатурного почерка и характера работы исследуемого или контролируемого субъекта с компьютерной мышью содержат информацию о психофизиологических состояниях оператора: нормальное, усталость, опьянение, возбужденное, расслабленное (сонное).

\subsubsection{Метод анализа клавиатурного почерка}
\textit{Клавиатурный почерк} --- это подвид поведенческой подгруппы аутентификации по неотчуждаемым признакам. \cite{keystroke}

Клавиатурный почерк определяется по времени между нажатиями клавиш. При снятии биометрического шаблона клавиатурного почерка измеряют время нажатия двух, трех или четырех последовательных клавиш, сохраняют его и на основе полученных значений строят математические модели для сравнения шаблонов нескольких пользователей. \cite{intrusionDetection}

 Для системы входными данными будут два биометрических шаблона --- эталонного и кандидата. Результат работы системы --- рейтинг доверия к биометрическому шаблону кандидата, который является критерием схожести двух переданных шаблонов.

Различают два вида распознавания клавиатурного почерка: распознавание на статическом тексте (пароль или известная кодовая фраза), распознавание при вводе псевдослучайного текста. \cite{keystroke}

Математическая задача распознавания на фиксированном тексте может быть формализована. Для ее выполнения потребуется \cite{keystroke}:
\begin{itemize}
\item собрать информацию о времени между нажатиями соседних клавиш в тексте;
\item сформировать из полученных данных вектор фиксированной размерности;
\item по кластерной модели или другой модели сравнения двух векторов сравнить сформированный вектор и вектор-эталон для этого же текста от этого же пользователя.
\end{itemize}

Для задачи распознавания клавиатурного почерка при вводе псевдослучайного текста не существует надежных моделей формирования пользовательского шаблона и вычисления рейтинга. Время работы подобных систем не позволяет в реальном времени оценить ситуацию и выдает результат через десятки минут, а память, занимаемая векторами, лишь продолжает расти. \cite{keystroke}

Одним из численных показателей, которые определяют качество биометрической системы, является \textit{ошибка первого рода (FRR, количество ложноотрицательных)} --- это вероятность ложного отказа в доступе. Данная ошибка имеет место при возникновении повреждения рук пользователя или ненормального психофизического состояния человека (усталость, алкогольное или наркотическое опьянение, приступ гнева).\cite{keystroke}

\textit{Ошибка второго рода (FAR, количество ложноположительных)} --- это вероятность ложного допуска. Данная ошибка имеет место при ситуации, когда заданные возможные отклонения от допустимых значений при распознавании пользователя были заданы неверно, либо же когда нарушитель сумел скопировать метрики поведения пользователя и обойти систему контроля. \cite{keystroke}

Для проектируемой системы важнейшую роль играет ошибка первого рода. Данная ошибка способна позволить распознать состояние усталости оператора в случае единоличного пользования системой.

\subsubsection{Метод анализа скорости печати и количества ошибок}
Изменение психофизического состояния пользователя приводит к увеличению времени совершения простейших действий. Контроль времени выполнения данных действий способен помочь определить ухудшение состояния оператора автоматизированного рабочего места и своевременно сообщить пользователю о необходимости соблюдения гигиены труда. К контролю действий оператора относят \cite{typeFatigue}:
\begin{itemize}
\item скорость печати (ввода) --- количество символов, введенных пользователем, разделенное на время, за которое данные символы были введены --- уменьшение количества символов, введенного в единицу времени, может символизировать ухудшение состояния пользователя;
\item динамика печати (ввода) --- время между нажатиями клавиш и временем их удержания --- увеличение интервала между нажатиями может символизировать ухудшение состояния пользователя;
\item частота возникновения ошибок при печати (вводе) --- увеличение количества совершаемых ошибок может символизировать ухудшение состояния пользователя.
\end{itemize}

\subsubsection{Метод анализа траекторий перемещения курсора мыши}
Компьютерная мышь используется для взаимодействия с оконным интерфейсом операционной системы и программ.

Особенности работы с мышью можно оценить, анализируя траектории передвижения курсора мыши по экрану между элементами интерфейса и среднее время выполнения передвижения. \cite{recognitionOfPsycho}

Оценка среднего времени перемещения курсора мыши между элементами интерфейса может быть выполнена с использованием адаптированным для данной задачи законом Фиттса \cite{fitts}:
\begin{equation}
\label{eq:fitts}
T = b \cdot \log_{2}\left(\frac{D}{W} + 1\right),
\end{equation}
\eqexplSetIntro{где}
\begin{eqexpl}[15mm]
\item{$b$} величина, зависящая от типичной скорости движения курсора мыши (отношение средней скорости движения мыши по экрану, осуществляемого субъектом, к установленному в операционной системе коэффициенту чувствительности мыши);
\item{$D$} дистанция перемещения курсора между элементами интерфейса ( в пикселях);
\item{$W$} ширина элемента интерфейса, к которому направляется курсор (в пикселях).
\end{eqexpl}

Адаптированный закон Фиттса связывает время, затраченное на движение к цели, с точностью движения и расстоянием до цели. Причем время перемещения курсора не совпадает с оценкой, выраженной в формуле \eqref{eq:fitts}, и отличается на некоторую величину, которую используют в качестве идентифицирующего признака. \cite{mouseMethod}

В качестве признаков могут быть использованы амплитуды первых десяти низкочастотных гармоник функции скорости перемещения курсора мыши по экрану:
\begin{equation}
\label{eq:harmonic}
V_{xy}(t) = \sqrt{\left(\left(x\left(t_{i+1}\right) - x\left(t_i\right)\right)^2 + \left(y\left(t_{i+1}\right)-y\left(t_i\right)\right)^2\right)^2},
\end{equation}
\eqexplSetIntro{где}
\begin{eqexpl}[15mm]
\item{$x$, $y$} координаты курсора;
\item{$t_i$} $i$-ый момент времени регистрации координат курсора (регистрация координат курсора зависит от производительности компьютера).
\end{eqexpl}

Разложение функции \eqref{eq:harmonic} производится с помощью быстрого преобразования Фурье, тем самым достигается нормирование участков пути курсора по времени \cite{mouseMethod}. Каждый участок приводится к длительности в $0.5$ секунд, амплитуды нормируются по энергии функции $V_{xy}(t)$, вычисляемой в соответствии с формулой:

\begin{equation}
\label{eq:normir}
E_s = \int\limits_{\infty}^{-\infty} A^2(\omega)\,dt,
\end{equation}
\eqexplSetIntro{где}
\begin{eqexpl}[15mm]
\item{$A(\omega)$} амплитуды гармоники с частотой $\omega$ функции $V_{xy}(t)$.
\end{eqexpl}

Операция нормирования приводит траектории перемещений курсора между элементами интерфейса к единому масштабу. Аналогичные операции осуществляются по отношению к функциям координат курсора $x(t)$ и $y(t)$ с выполнением предварительного перевода данных функций в систему координат, где точкой начала координат является центр элемента интерфейса, к которому двигался курсов, причем ось абсцисс будет расположена в направлении к центру данного элемента. Данные действия обусловлены необходимостью избавиться от наклона линий, связывающих элементы интерфейса, относитльено исходной координатной плоскости. \cite{mouseMethod}

\subsubsection{Метод анализа использования координатного устройства мышь с использованием модели искусственной нейронной сети}
В качестве главных факторов биометрической информации в данном методе выбираются параметры применения клавиатуры и мыши. \cite{neuroFatigue}

Для сбора и хранения особенностей ввода управляющих последовательностей применяется логирование клавиатурного буфера машины. При любом использовании клавиатуры сохраняются задержки между нажатиями, продолжительности нажима отдельных клавиш, а также продолжительность набора устойчивого сочетания. В качестве сохраняемой информации об использовании мыши и иных координатных устройств фиксируются изменения положений, скорости и времени передвижения курсора. \cite{neuroFatigue}

Анализ сохраненных данных проводится с использованием специальной искусственной нейронной сети. \cite{neuroFatigue}

Основной задачей нейронной сети в данном методе является разбиение поступающих данных на два кластера: случаи санкционированного доступа и несанкционированного доступа. При этом исследуемый показатель не является статическим, в связи с чем нейронная сеть работает в динамическом режиме, переобучаясь по мере изменения поведенческих особенностей пользователя. \cite{neuroFatigue}

После обучения нейронной сети, с ее помощью исследуется поток данных о работе текущего пользователя. Из-за возможности изменения поведения ввода человека, с некоторой долей вероятности модель может не может однозначно определить правомерность доступа. В подобных ситуациях требуется принять решение о том, следует ли переобучить модель в связи с изменением анализируемых параметров или запретить доступ к системе. Для решения данной проблемы используется метод нечеткого вывода Мамдани. \cite{neuroFatigue}

\subsubsection{Метод анализа внешнего состояния пользователя}
Основные исследования в области использования видео-изображений для определения опасных состояний усталости проводятся для реализации систем распознавания усталости водителя.

Признаки состояний ослабленного внимания и усталости у водителя характеризуется следующими наблюдаемыми параметрами \cite{videoMethod}:
\begin{itemize}[leftmargin=1.6\parindent]
\item поворот головы влево/вправо по отношению к туловищу;
\item наклон головы вперед относительно туловища;
\item продолжительность моргания век;
\item частота моргания век;
\item степень открытости рта человека (признаки зевоты).
\end{itemize}

Работа системы строится на использовании технологий машинного обучения и предварительно обученных на наборах данных моделей. Детектирование опасных состояний в поведении повышает точность и полноту определения ослабленного внимания и усталости --- накопление, анализ и обработка информации, поступающей в облачный сервис и формирующей историю вождения, приводят к созданию новых моделей поведения того или иного пользователя, вычислять характерные параметры, оказывающие влияние на эффективность работы схем распознавания опасных состояний. Таким образом высчитываются не только пороговые параметры для каждого водителя индивидуально, но и поддерживается надежность обновляемых параметров для новых пользователей, чьи параметры пока не известны. \cite{videoMethod}

Применение рассматриваемого метода, в комплексе с уже рассмотренными, для определения состояния усталости оператора автоматизированного рабочего места потребует больших вычислительных ресурсов серверной части программного комплекса. Кроме того, со стороны клиента потребуется беспрерывное соединение с сетью Интернет, увеличение ресурса оперативной памяти и современная веб-камера с приемлемым качеством фото- и видео-съемки.

\subsubsection{Метод анализа речевых характеристик пользователя}
Микрофон позволяет регистрировать аудиопоток, исходящий от пользователя и его окружения.

Исследования показали, что признаки голоса наилучшим образом позволяют распознавать усталость или расслабленное (сонное) состояние диктора. \cite{recognitionOfPsycho} 

Симптоматика усталости заключается в проявлениях слабости, раздражительности, расстройствах сна и вегетативных нарушениях. \cite{medObozr}

Степень раздражительности может быть оценена по шкале Рэймонда Новако. Тест Новако содержит в себе 25 вопросов, ответы на которые позволяют определить показатель раздражительности. При интерпретации результатов показатель в 0--45 баллов определяет низкую степень раздражительности, 46--55 баллов --- раздражительность ниже средней, 56--75 баллов --- среднюю степень раздражительности, 76--85 баллов --- раздражительность выше средней и 86--100 баллов --- высокую степень раздражительности. \cite{novako}

Проявления слабости, расстройства сна и вегетативные нарушения могут быть установлены с использованием клинических методов.

% Таким образом, с использованием распознавания агрессии, точность верного распознавания которой составляет порядка 90\% \cite{recognitionOfPsycho}, поток информации о состоянии оператора может быть дополнен также и голосовыми характеристиками.

Наиболее полную информацию о внутреннем психоэмоциональном состоянии человека может дать анализ его связной речи \cite{voiceMethod}:
\begin{itemize}[leftmargin=1.6\parindent]
\item расстановка логических ударений;
\item скорость произнесения слов;
\item конструкция фразы;
\item неуверенный или неверный подбор слов;
\item обрывание фраз на полуслове;
\item изменение слов;
\item появление слов-паразитов;
\item исчезновение пауз. 
\end{itemize}

Разработки в сфере анализа напряжения голоса тестируются в судебной психологии для выявления лжи и обмана посредством обнаружения микротремора. Однако целесообразность и надежность использования данного метода до сих пор является предметом дискуссий в силу того, что успех в выявлении лжи зависит от опыта эксперта. Кроме того, было доказано, что невиновные люди, чья невиновность оспаривается, проявляют не меньше стресса, чем виновные, увеличивая риск появления ложных срабатываний. Данные проблемы в практике привели к тому, что на сегодняшний день данные разработки рассматриваются как многообещающий инструмент обнаружения стресса, однако проблема межиндивидуальных различий пока еще остается нерешенной. \cite{whyMicrophoneIsShit}

\subsubsection{Метод анализа виброакустических шумов при наборе текста или использовании мыши}
Данный метод предполагает использование виброакустических датчиков, установленных непосредственно на рабочем столе пользователя, данные с которых вводятся в звуковую карту. \cite{acousticFatigue}

Для идентификации оператора из записей виброакустического сигнала при наборе произвольного текста удаляются паузы между нажатиями и отпусканиями клавиш клавиатуры. Затем очищенный исследуемый сигнал разбивается на перекрывающиеся интервалы, длины которых определяются идентификацией оператора с минимальной погрешностью. Основное требование для векторов идентификации --- устойчивость для всех пользователей в системе распознавания. Таким образом формируется база виброакустических сигналов, возникающих при наборе каждым из пользователей системы. \cite{acousticFatigue}

Для идентификации оператора с использованием манипулятора мышь в качестве векторов признаков используются относительные длительности периодов активности пользователя, которые вычисляются на участках виброакустического сигнала, сопоставленных с записями в журнале событий мыши. \cite{mouseFatigue}

Для каждого оператора обучается нейронная сеть для его идентификации и затем используется при работе системы. \cite{acousticFatigue}

В качестве оптимизации формирования обучающей выборки предлагается использование алгоритма выделения фрагментов, которые содержат виброакустический сигнал нажатия или отпускания клавиш на основе ошибки линейного предсказания и последующего удаления единичных максимумов. \cite{acousticFatigue}

\subsection{Биофизические факторы, позволяющие определить усталость}

\subsubsection{Частота пульса и возраст сосудистой системы}
Согласно исследованиям \cite{stressInvestigation}, при кратковременном стрессовом воздействии было выявлено три показателя, которые подвергаются изменениям: частота пульса, возраст сосудистой системы и индекс стресса.

Частота пульса, в среднем, до стрессового воздействия составляла $77.42 \pm 4.12$ ударов в минуту, после --- $99.67 \pm 5.54$ ударов в минуту с уровнем достоверности $p = 0.004$. \cite{stressInvestigation}

\textit{Возраст сосудистой системы} --- это параметр, определяющий биологический возраст индивида, то есть изношенность его организма. \cite{ageOfVascularSystem} 

Данный параметр, в среднем, до стрессового воздействия был равен $29 \pm 1.9$ лет, после --- $34.5 \pm 1.69$ лет с уровнем достоверности $p = 0.041$. \cite{stressInvestigation}

\subsubsection{Индекс Баевского}
\textit{Индекс стресса (индекс Баевского)} характеризует вариабельность сердечного ритма и состояние центров регуляции сердечно-сосудистой системы. Норма данного индекса для человека находится в диапазоне от 50 до 150 единиц. Увеличение данного показателя до значений от 150 до 500 может говорить о наличии физических нагрузок или хронической усталости. Увеличение индекса до значений в диапазоне от 500 до 900 говорит о наличии существенного психологического и эмоционального стресса, либо о психофизиологическом переутомлении или стенокардии. Превышение индексом значения 900 единиц свидетельствует о нарушении регуляторных механизмов или предынфарктном состоянии. \cite{stressIndex}

Индекс Баевского может быть рассчитан по следующей формуле \cite{baevskiy}:
\begin{equation}
\label{eq:baevskiy}
I = \frac{AM_o}{2 M_o \Delta X},
\end{equation}
\eqexplSetIntro{где}
\begin{eqexpl}[15mm]
\item{$M_o$} это наиболее часто встречающееся в динамическом ряде значение кардиоинтервала (мода);
\item{$AM_o$} это число кардиоинтервалов, соответствующих значению моды (амплитуда моды), в процентах от объему выборки;
\item{$\Delta X$} вариационный размах среднего значения продолжительности\\сердечного цикла, показатель деятельности контура автономной регуляции ритма сердца, который целиком связан с дыхательными колебаниями тонуса блуждающих нервов, в секундах.
\item{$M_o \Delta X$} отражает степень вариативности значений кардиоинтервалов в исследуемом динамическом ряду.
\end{eqexpl}

В исследовании данный параметр, в среднем, до стрессового воздействия был равен $75.27 \pm 11.84$ единиц, после --- $334.55$ единиц с уровнем достоверности $p = 0.001$.

Частота сердечных сокращений является включением для индекса стресса, из чего следует факт того, что в системе отсутствует потребность в анализе данной характеристики отдельно. Возраст сосудистой системы --- параметр, определяемый сложной медицинской техникой. В силу изложенных фактов, в дальнейшем в систему в качестве характеристики для определения стресса войдет лишь индекс Баевского.

В качестве устройства для снятия индекса, основанного на вариабельности сердечного ритма, могут быть предложены смарт-часы с поддержкой данной функции. Модуль программного обеспечения, сохраняющий данные характеристики, должен записывать ее значения в базу данных с использованием интерфейса, предоставляемым выбранной моделью.


\subsection{Формализация требований к методу систематического распознавания усталости на автоматизированном рабочем месте}
На текущий момент не каждое автоматизированное рабочее место может включать в себя все рассмотренные внешние устройства взаимодействия пользователя с системой. Данное обстоятельство указывает на требование возможности работы системы в условиях отсутствия тех или иных периферийных устройств.

К требованиям также отнесена потребность в сохранении ресурсов персонального компьютера автоматизированного рабочего места в силу отсутствия стандартизации конфигурации технической составляющей компьютера.

Важно отметить, что в системе требуется разграничение личностей пользователей в силу индивидуальности характеристик каждого.

\subsubsection*{Вывод}
%Были рассмотрены понятия усталости, хронической усталости и стресса, описаны стадии адаптационного синдрома и причины профессионального стресса.

%Была определена цель метода систематического распознавания усталости на автоматизированном рабочем месте: предупреждение наступления стадии истощения посредством регистрации наступления стадии тревоги или стадии тревоги и затем устойчивости.

%Были рассмотрены следующие внешние устройства, позволяющие определить усталость пользователя автоматизированного рабочего места: клавиатура, координатное устройство типа мышь, веб-камера, микрофон.

%Был рассмотрен метод распознавания усталости пользователя по клавиатурному почерку, который определяется по времени между нажатиями клавиш. Описана математическая задача распознавания почерка на фиксированном тексте, которая включила в себя:
%\begin{itemize}[leftmargin=1.6\parindent]
%\item сбор информации о времени между нажатиями соседних клавиш в тексте;
%\item формирование из полученных данных вектора фиксированной размерности;
%\item сравнение сформированного вектора и вектора-эталона для того же текста и для того же пользователя. 
%\end{itemize}

%Было определено, что ошибка первого рода при решении задачи распознавания почерка играет важнейшую роль для проектируемой системы, так как она способна позволить распознать состояние усталости оператора.

%Был рассмотрен метод распознавания усталости пользователя по разнице скорости перемещения курсора и оценки среднего времени перемещения курсора мыши между элементами интерфейса, которая определяется по адаптированному к задаче закону Фиттса.

%Был рассмотрен метод распознавания усталости с использованием изображений, получаемых от веб-камеры, который строится на использовании технологий машинного обучения и предварительно обученных на наборах данных моделей.

%Был рассмотрен метод распознавания усталости с использованием аудио-потока, построенного на анализе связной речи человека. Однако проведенная аналитическая работа показала, что целесообразность и надежность использования данного метода подвергается сомнению в силу высокой вероятности ложных срабатываний и межиндивидуальных различий.

%Были формализованы требования к методу систематического распознавания усталости на автоматизированном рабочем месте, которые включили в себя:
%\begin{itemize}[leftmargin=1.6\parindent]
%\item возможность работы системы в условиях отсутствия тех или иных периферийных устройств;
%\item рациональное использование ресурсов персонального компьютера автоматизированного рабочего места;
%\item разграничение личностей пользователей. 
%\end{itemize}

%Был построен метод систематического распознавания усталости на автоматизированном рабочем месте. 

%Рассмотрены биофизические факторы, позволяющие определить усталость пользователя: частота пульса, возраст сосудистой системы, индекс стресса (индекс Баевского). В качестве задействованной в системе характеристики был выбран индекс стресса в силу того, что частота сердечных сокращений является включением для данного индекса, а возраст сосудистой системы --- параметр, определяемый сложной медицинской техникой.

%Было определено, что для определения индекса Баевского будут использоваться смарт-часы с поддержкой данной функции.

%Была выбрана включаемый в метод характеристика, получаемая с использованием внешнего устройства, --- ошибка первого рода для задачи распознавания клавиатурного почерка на статическом тексте. Данный выбор обусловлен следующими факторами:
%\begin{itemize}[leftmargin=1.6\parindent]
%\item координатное устройство типа мышь может быть использовано лишь в качестве идентификационного средства;
%\item применение метода распознавания усталости с использованием веб-камеры может потребовать больших вычислительных ресурсов серверной части программного комплекса, а также постоянное соединение с сетью Интернет со стороны клиента;
%\item метод распознавания усталости и стресса с использованием микрофона позволяет определить лишь проявления слабости, которые не могут однозначно указывать на наступление стадии истощения, адаптации или тревоги. 
%\end{itemize}

%Формализованный метод систематического распознавания усталости на автоматизированном рабочем месте включил в себя:
%\begin{itemize}[leftmargin=1.6\parindent]
%\item хранение и анализ данных, получаемых от клавиатуры;
%\item хранение и анализ данных об индексе стресса субъекта. 
%\end{itemize}


\pagebreak