\section{Аналитический раздел}

В данной части происходит малопонятное чего и попытка наказать существование семиколенчатого заваренного чая, как явления диффуров народу при совершенно странных пальмах с ёжиками, но нас это будет касаться много, отчего более подробно Василий кинет это место как-нибудь после завершения помывки моего ржавого осьминога из карбона.

\subsection{Устройства взаимодействия пользователя АРМ с системой}
\textit{Внешние устройства (периферийные устройства)} --- устройства ввода-вывода, распечатки, хранения и передачи информации, связанные функционально с центральным процессором в соответствии со структура ЭВМ (или системы ЭВМ). \cite{encDic}

К внешним устройствам, которые являются устройствами ввода, то есть органами управления персональным компьютером, относят:
\begin{itemize}[leftmargin=1.6\parindent]
\item[1)] Клавиатуру;
\item[2)] Мышь;
\item[3)] Графический планшет;
\item[4)] Веб-камеру;
\item[5)] Микрофон;
\item[6)] Игровой манипулятор.
\end{itemize}

Из указанных выше устройств в рассмотрение не войдут:
\begin{itemize}[leftmargin=1.6\parindent]
\item[1)] Графический планшет;
\item[2)] Игровой манипулятор.
\end{itemize}

Данное решение связанно с тем, что использование подобного рода устройств указывает на особый род работы. В требованиях к разрабатываемой системе указывается возможность её использования для операторов автоматизированных рабочих мест в наиболее распространенных  конфигурациях (например, для офиса), поэтому использование в решении поставленной задачи информации, получаемой с графических планшетов и игровых манипуляторов, не является целесообразным.

Согласно проведённым исследованиям \cite{recognitionOfPsycho} признаки голоса, клавиатурного почерка и характера работы исследуемого или контролируемого субъекта с компьютерной мышью содержат информацию о психофизиологических состояниях оператора: нормальное, усталость, опьянение, возбужденное, расслабленное (сонное).

\subsubsection{Клавиатура}
\textit{Клавиатура} --- одно из наиболее часто используемых внешних устройств для взаимодействия пользователя с персональным компьютером.

\textit{Клавиатурный почерк} --- это подвид поведенческой подгруппы аутентификации по неотчуждаемым признакам. \cite{keystroke}

Клавиатурный почерк определяется по времени между нажатиями клавиш. При снятии биометрического шаблона клавиатурного почерка измеряют время нажатия двух, трех или четырех последовательных клавиш, сохраняют его и на основе полученных значений строят математические модели для сравнения шаблонов нескольких пользователей. \cite{intrusionDetection} Для системы входными данными будут два биометрических шаблона --- эталонного и кандидата. Результат работы системы --- рейтинг доверия к биометрическому шаблону кандидата, который является критерием схожести двух переданных шаблонов.

Различают два вида распознавания клавиатурного почерка: распознавание на статическом тексте (пароль или известная кодовая фраза), распознавание при вводе псевдослучайного текста. \cite{keystroke}

Математическая задача распознавания на фиксированном тексте может быть формализована. Для её выполнения потребуется \cite{keystroke}:
\begin{itemize}
\item[1)] Собрать информацию о времени между нажатиями соседних клавиш в тексте;
\item[2)] Сформировать из полученных данных вектор фиксированной размерности;
\item[3)] По кластерной модели или другой модели сравнения двух векторов сравнить сформированный вектор и вектор-эталон для этого же текста от этого же пользователя.
\end{itemize}

Для задачи распознавания клавиатурного почерка при вводе псевдослучайного текста не существует надежных моделей формирования пользовательского шаблона и вычисления рейтинга. Время работы подобных систем не позволяет в реальном времени оценить ситуацию и выдает результат через десятки минут, а память, занимаемая векторами, лишь продолжает расти. \cite{keystroke}

Одним из численных показателей, которые определяют качество биометрической системы, является \textit{ошибка первого рода (FRR, количество ложноотрицательных)} --- это вероятность ложного отказа в доступе. Данная ошибка имеет место при возникновении повреждения рук пользователя или ненормального психофизического состояния человека (усталость, алкогольное или наркотическое опьянение, приступ гнева).\cite{keystroke}

\textit{Ошибка второго рода (FAR, количество ложноположительных)} --- это вероятность ложного допуска. Данная ошибка имеет место при ситуации, когда заданные возможные отклонения от допустимых значений при распознавании пользователя были заданы неверно, либо же когда нарушитель сумел скопировать метрики поведения пользователя и обойти систему контроля. \cite{keystroke}

Для проектируемой системы важнейшую роль играет ошибка первого рода. Данная ошибка способна позволить распознать состояние усталости оператора в случае единоличного пользования системой.

\subsubsection{Координатное устройство типа мышь}
Компьютерная мышь используется для взаимодействия с оконным интерфейсом операционной системы и программ.

Особенности работы с мышью можно оценить, анализируя траектории передвижения курсора мыши по экрану между элементами интерфейса и среднее время выполнения передвижения. \cite{recognitionOfPsycho}

Оценка среднего времени перемещения курсора мыши между элементами интерфейса выполняется с использованием адаптированной для данной задачи формулы \ref{eq:fitts} закона Фиттса \cite{fitts}. Данный закон связывает время движения к наблюдаемой цели с точностью движения и с расстоянием до наблюдаемой цели. Чем дальше или точнее выполняется движение руки субъекта, тем больше коррекции необходимо для его выполнения, и соответственно, больше времени требуется субъекту для внесения этой коррекции. Фактическое время перемещения не должно совпадать с оценкой, вычисляемой по формуле \ref{eq:fitts}, а должно отличаться на величину $\Delta T$, которую используют как один из идентифицирующих признаков. \cite{mouseMethod}

\begin{equation}
\label{eq:fitts}
T = b \cdot \log_{2}\left(\frac{D}{W} + 1\right),
\end{equation}

\eqexplSetIntro{где:}
\begin{eqexpl}[15mm]
\item{$b$} величина, зависящая от типичной скорости движения курсора мыши (отношение средней скорости движения мыши по экрану, осуществляемого субъектом, к установленному в операционной системе коэффициенту чувствительности мыши);
\item{$D$} дистанция перемещения курсора между элементами интерфейса ( в пикселях);
\item{$W$} ширина элемента интерфейса, к которому направляется курсор (в пикселях).
\end{eqexpl}

Также в качестве признаков предлагается использовать амплитуды первых десяти низкочастотных гармоник функции скорости перемещения курсора мыши по экрану $V_{xy}(t)$, вычисляемой по формуле \ref{eq:harmonic} разложение функции производится с помощью быстрого преобразования Фурье, тем самым достигается нормирование участков пути курсора по времени. Каждый участок приводится к длительности в $0.5$ секунд. Амплитуды нормируются по энергии функции $V_{xy}(t)$, вычисляемой в соответствии с формулой \ref{eq:normir}, данная операция осуществляется для того, чтобы привести все траектории перемещений курсора между элементами интерфейса к единому масштабу. Аналогичные операции осуществляются по отношению к функциям координат курсора $x(t)$ и $y(t)$, однако прещдварительно данные функции переводятся в другую систему координат, где началом координат является центр элемента интерфейса, на который было произведено нажатие, ось абсцисс располагается в направлении центра элемента интерфейса, по отношению к которому произхводится перемещение курсора. Это необходимо выполнять, чтобы избавиться от наклона линий, связывающих элементы интерфейса, относительно исходной координатной плоскости (то есть зависимости координат от угла наклона). \cite{mouseMethod}

\begin{equation}
\label{eq:harmonic}
V_{xy}(t) = \sqrt{((x(t_{i+1}) - x(t_i))^2 + (y(t_{i+1})-y(t_i))^2)^2},
\end{equation}

\eqexplSetIntro{где:}
\begin{eqexpl}[15mm]
\item{$x$, $y$} координаты курсора;
\item{$t_i$} $i$-ый момент времени регистрации координат курсора (регистрация координат курсора зависит от производительности компьютера).
\end{eqexpl}

\begin{equation}
\label{eq:normir}
E_s = \int\limits_{\infty}^{-\infty} A^2(\omega)\,dt,
\end{equation}

\eqexplSetIntro{где:}
\begin{eqexpl}[15mm]
\item{$A(\omega)$} амплитуды гармоники с частотой $\omega$ функции $V_{xy}(t)$.
\end{eqexpl}

\subsubsection{Веб-камера}
Основные исследования в области использования видео-изображений для определения опасных состояний усталости проводятся для реализации систем распознавания усталости водителя.

Признаки состояний ослабленного внимания и усталости у водителя характеризуется следующими наблюдаемыми параметрами \cite{videoMethod}:
\begin{itemize}[leftmargin=1.6\parindent]
\item[1)] Поворот головы влево/вправо по отношению к туловищу;
\item[2)] Наклон головы вперед относительно туловища;
\item[3)] Продолжительность моргания век;
\item[4)] Частота моргания век;
\item[5)] Степень открытости рта человека (признаки зевоты).
\end{itemize}

В процессе работы системы происходит накопление и обработка статистических данных от всех водителей. В таком случае технологии машинного обучения и предварительно обученные на наборах данных модели позволяют повысить точность и корректность в распознавании как состояния ослабленного внимания, так и усталости водителя. Детектирование опасных состояний в поведении водителя в онлайн и офлайн режимах позволяет повысить точность и полноту детектирования ослабленного внимания и усталости водителя в кабине транспортного средства: накопление, анализ и обработка информации, поступающей в облачный сервис и, таким образом, формирующей всю историю вождения всех водителей, позволяет создавать новые модели поведения того или иного водителя, проводить эксперименты, вычислять характерные параметры вождения, оказывающие влияние на эффективность работы схем распознавания опасных состояний. Таким образом высчитываются не только пороговые параметры для каждого водителя индивидуально, но и поддерживается надёжность обновляемых параметров для водителей, параметры которых пока не известны. \cite{videoMethod}

Применение рассматриваемого метода, в комплексе с уже рассмотренными, для определения состояния усталости оператора автоматизированного рабочего места потребует больших вычислительных ресурсов серверной части программного комплекса. Кроме того, со стороны клиента потребуется беспрерывное соединение с сетью Интернет, увеличение ресурса оперативной памяти и современная веб-камера с приемлемым качеством фото- и видео-съёмки.

\subsubsection{Микрофон}
Микрофон позволяет регистрировать аудиопоток, исходящий от пользователя и его окружения.

Исследования \cite{recognitionOfPsycho} показали, что признаки голоса наилучшим образом позволяют распознавать усталость или расслабленное (сонное) состояние диктора. 

Согласно статье \cite{medObozr} симптоматика усталости заключается в проявлениях слабости, раздражительности, расстройствах сна и вегетативных нарушениях. К симптомам раздражительности относят:
\begin{itemize}[leftmargin=1.6\parindent]
\item[1)] Гневливость;
\item[2)] Повышенная возбудимость;
\item[3)] Придирчивость;
\item[4)] Брюзгливость;
\item[5)] Недовольство по любому поводу и без явного повода.
\end{itemize}

Таким образом, с использованием распознавания агрессии, точность верного распознавания которой составляет порядка 90\% \cite{recoginitionOfPsycho}, поток информации о состоянии оператора может быть дополнен также и голосовыми характеристиками.

В общем, если дальше утвердится такая форма, то опишу методы отсюда вот\cite{voiceMethod}

\subsubsection*{Вывод}
Цель данной работы --- создание благоприятных условий на рабочих местах. Решение задачи установления зависимости работоспособности и физиологического состояния работника может позволить сохранить здоровье и работоспособность трудящихся, решить проблему повышения эффективности работы, заболеваемости на высоконагруженных трудовых местах и иных вопросов здравоохранения.


\pagebreak