\section*{ВВЕДЕНИЕ}
\addcontentsline{toc}{section}{ВВЕДЕНИЕ}

Согласно исследованиям \cite{burnout}, на момент 2019 года 28\% сотрудников постоянно или достаточно часто чувствовали себя угнетёнными под давлением рабочих обязанностей, а 48\% страдали синдромом эмоционального выгорания время от времени.

На сегодняшний день проблема эмоционального выгорания касается не только самих работников, но и компаний, которые при утрате контроля над ситуацией вынуждены увольнять сотрудников под предлогом неисполнения ими обязанностей. \cite{CompanyProblem}

Причиной синдрома может быть и физическое, и эмоциональное истощение вследствие увеличения нагрузки на работе, а также количества возлагаемых обязанностей на сотрудника. \cite{Prichini}

Синдром хронической усталости также является одним из факторов, понижающих работоспособность сотрудников. Несмотря на то, что этиология данного заболевания до конца не раскрыта, одним из возможных факторов появления данного синдрома приписывают высокой нагрузке как умственной, так и физической. \cite{SHU}

Усталость негативно влияет на производительность труда, а также на психологическое и физическое состояние человека. В условиях современной цифровизации медицины становится реальным учитывать индивидуальные особенности организма, управление его работоспособностью и проведение профилактики проявления вышеописанных синдромов, приводящих к неутешительным последствиям.

\pagebreak