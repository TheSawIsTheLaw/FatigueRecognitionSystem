\section{Технологическая часть}
\subsection{Средства реализации программного обеспечения}
При написании программного продукта был использован язык программирования Kotlin \cite{Kotlin}.

Данный выбор обусловлен следующими факторами:
\begin{itemize}[leftmargin=1.6\parindent]
\item возможность запуска программного кода на любом устройстве, поддерживающем Java Virtual Machine;
\item большое количество актуализируемой справочной литературы, связанной как я языком программирования Java, так и Kotlin;
\item возможность интеграции программного кода в приложения для ОС Android.
\end{itemize}

При написании программного продукта использовалась среда разработки IntelliJ IDEA. Данный выбор обусловлен тем, что Kotlin является продуктом компании JetBrains, поставляющей данную среду разработки.

\subsection{Выбор СУБД}

\subsubsection{Базы данных временных рядов}
Базы данных временных рядов отличаются от статических баз данных тем, что содержат записи, в которых некоторые из атрибутов ассоциируются с временными метками. В качестве таких записей могут выступать данные мониторинга, биржевые данные о торгах или транзакции продаж. \cite{bdvrAnomalies}

\paragraph{InfluxDB}
InfluxDB - это база данных временных рядов, предназначенная для обработки высокой нагрузки записи и запросов.

Основным назначением является хранение больших объемов данных с метками времени. Например, данные мониторинга, метрики приложений и данные датчиков IoT (Internet of Things, интернет вещей).

В традиционной реляционной базе данных данные хранятся до тех пор, пока вы не решите их удалить. Учитывая сценарии использования баз данных временных рядов, можно не хранить данные слишком долго: это или слишком дорого, или данные со временем теряют актуальность. \cite{ryadi}

Системы вроде InfluxDB могут удалять данные спустя определённое время, используя концепцию, называемую политикой хранения. Также имеется возможность выполнять непрерывные запросы к оперативным данным для выполнения определённых операций. \cite{ryadi}

\paragraph{OpenTSDB}
OpenTSDB состоит из демона временных рядов, а также набора утилит командной строки. Взаимодействие с OpenTSDB в первую очередь достигается путем запуска одного или нескольких независимых демонов.

Демон использует базу данных с открытым исходным кодом HBase или службу Google Bigtable для хранения и получения данных временных рядов. Схема данных высоко оптимизирована для быстрого объединения аналогичных временных рядов, чтобы минимизировать пространство хранения. Пользователям никогда не требуется прямой доступ к базовому хранилищу. Можно общаться с демоном через протокол telnet, HTTP API или простой встроенный графический интерфейс.

\subsubsection{Реляционные базы данных}
Реляционная база данных --- это организованный по реляционной модели набор таблиц, в которых каждая ячейка этих таблиц имеет некоторое соответствующее описание. \cite{relationki}

Использование реляционной модели предполагает возможность идентификации элементов по совокупности уникальных идентификаторов: имя столбца, первичный ключ. Для построения логической связи между строками и ячейками разных таблиц используются внешние ключи. \cite{relationki}

Среди подобных СУБД, которые основаны на реляционных базах данных выделяются: Oracle, PostgreSQL.

Каждая из указанных СУБД имеет некоторые отличительные особенности. Так, например, PostgreSQL поддерживает вставки кода, написанного на языке программирования Python, в тело процедуры. Однако выделить среди данных особенностей важных для данной работы не представляется возможным.

\subsubsection*{Вывод}
Базы данных временных рядов являются наиболее подходящими для решаемой задачи, так как они нацелены на хранение, извлечение и анализ большого количества статистических данных, в которых имеются временные метки.

Для организации хранения данных будет использоваться СУБД InfluxDB, так как она является одной из самых популярных, среди известных баз данных временных рядов, а также по той причине, что поддержка данной СУБД всё ещё не прекращена на сегодняшний день.

\subsection{Выбор алгоритма кластеризации}
В качестве используемого алгоритма кластеризации был выбран метод c-средних в силу того, что число кластеров заранее известно, а также задача рассматривает установку соответствия некоторого объекта (например, значения скорости печати) набора вещественных значений, показывающих степень отношения объекта к кластерам.

\subsection{Данные для кластеризации}
В качестве данных для кластеризации используются действия оператора автоматизированного рабочего места, производимые с использованием клавиатуры и мыши. Данные действия логируются с использованием программного обеспечения, а затем направляются в базу данных для возможности миграции определенной модели поведения на другое автоматизированное место.

Действия пользователя, участвующие в построении модели, соотносятся с временем реакции, которое фиксируется каждые 20 минут, причем по времени реакции определяется, в каком состоянии в текущий момент времени находится организм оператора.

\subsection{Сведения о модулях}
Программное обеспечение состоит из модулей логирования действий оператора, обработки и анализа данных.

\subsubsection{Модуль логирования действий оператора}
Данный модуль предназначен для записи информации о действиях оператора.

Внешние зависимости модуля:

\begin{itemize}[leftmargin=1.6\parindent]
\item Java Swing \cite{swing} --- библиотека легковесных компонентов для реализации оконного интерфейса приложения;
\item JNativeHook \cite{jnativehook} --- библиотека, предоставляющая средства перехвата прерываний, поступающих от клавиатуры и мыши.
\end{itemize}

Модуль состоит из четырех пакетов.

\paragraph{Пакет window}
Данный пакет включает в себя абстрактный класс Window, предоставляющий родительский класс с определенными свойствами для всех окон реализуемого программного обеспечения.

\lstinputlisting[
	caption={Файл window.kt},
	label={lst:window},
	language=kotlin
]{../BigBrother/src/main/kotlin/window/window.kt}

\paragraph{Пакет bigBrother}
Данный пакет включает в себя класс BigBrotherWindow, который является реализацией класса Window и определяет функционал главного экрана приложения.

\paragraph{Пакет loggers}
Данный пакет включает в себя пакеты реализаций логирующих классов: KeyLogger (нажатия на клавиши клавиатуры), MouseLogger (нажатия на клавиши мыши и движения данного устройства), ReactionLogger (результаты пройденных тестов на реакцию).

Также в данном пакете предоставлена реализация класса ReactionTestWindow, предоставляющая интерфейс и логику определения реакции пользователя по нажатию на кнопку, появляющуюся в случайные моменты времени (от 2 до 10 секунд).

Каждый логирующий класс локально создает текстовый файл, в который записывает в определенном формате собранные данные. Для исключения попыток изменения файла конкурирующими потоками в каждом из них представлена реализация очереди записи, которая переносится в файл при достижения размера в сотню записей, либо по завершению работы приложения.

\paragraph{Пакет random}
Данный пакет включает в себя класс, реализованного по паттерну ``Одиночка'', предоставляющий доступ к классу Random, инициализированного отложенно. Данный класс используется преимущественно для получения данных о реакции пользователя.

Также в модуле определен файл main.kt, который является точкой входа в приложения.

На рисунке \ref{fig:loggingUml} предоставлена диаграмма классов модуля.
\begin{figure}[H]
	\centering
	\includegraphics[width=\textwidth]{img/logger.pdf}
	\caption{Диаграмма классов модуля логирования.}
	\label{fig:loggingUml}
\end{figure}

%\begin{itemize}[leftmargin=1.6\parindent]
%\item bigBrother:
%	\begin{itemize}[leftmargin=1.6\parindent]
%	\item BigBrotherWindow;
%	\end{itemize}
%\item loggers:
%	\begin{itemize}[leftmargin=1.6\parindent]
%	\item keyLogger:
%		\begin{itemize}[leftmargin=1.6\parindent]
%		\item KeyLogger;
%		\end{itemize}
%	\item mouseLogger:
%		\begin{itemize}[leftmargin=1.6\parindent]
%		\item MouseLogger;
%		\end{itemize}
%	\item reactionTest:
%		\begin{itemize}[leftmargin=1.6\parindent]
%		\item ReactionLogger;
%		\item ReactionLoggerTestWindow;
%		\end{itemize}
%	\item Logger;
%	\end{itemize}
%\item random:
%	\begin{itemize}[leftmargin=1.6\parindent]
%	\item BBRandom;
%	\end{itemize}
%\item window:
%	\begin{itemize}[leftmargin=1.6\parindent]
%	\item Window;
%	\end{itemize}
%\item main.
%\end{itemize}

\subsubsection{Модуль обработки данных}

Модуль обработки данных включает в себя два пакета: парсинга данных и приведения данных.

Модуль парсинга данных...

Модуль приведения данных...

\subsubsection{Модуль анализа данных}


\subsection{Развертывание системы}

