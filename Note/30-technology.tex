\section{Технологическая часть}
\subsection{Средства реализации программного обеспечения}
При написании программного продукта был использован язык программирования Kotlin \cite{Kotlin}.

Данный выбор обусловлен следующими факторами:
\begin{itemize}[leftmargin=1.6\parindent]
\item возможность запуска программного кода на любом устройстве, поддерживающем Java Virtual Machine;
\item большое количество актуализируемой справочной литературы, связанной как я языком программирования Java, так и Kotlin;
\item возможность интеграции программного кода в приложения для ОС Android.
\end{itemize}

При написании программного продукта использовалась среда разработки IntelliJ IDEA. Данный выбор обусловлен тем, что Kotlin является продуктом компании JetBrains, поставляющей данную среду разработки.

\subsection{Выбор СУБД}

\subsubsection{Базы данных временных рядов}
Базы данных временных рядов отличаются от статических баз данных тем, что содержат записи, в которых некоторые из атрибутов ассоциируются с временными метками. В качестве таких записей могут выступать данные мониторинга, биржевые данные о торгах или транзакции продаж. \cite{bdvrAnomalies}

\paragraph{InfluxDB}
InfluxDB - это база данных временных рядов, предназначенная для обработки высокой нагрузки записи и запросов.

Основным назначением является хранение больших объемов данных с метками времени. Например, данные мониторинга, метрики приложений и данные датчиков IoT (Internet of Things, интернет вещей).

В традиционной реляционной базе данных данные хранятся до тех пор, пока вы не решите их удалить. Учитывая сценарии использования баз данных временных рядов, можно не хранить данные слишком долго: это или слишком дорого, или данные со временем теряют актуальность. \cite{ryadi}

Системы вроде InfluxDB могут удалять данные спустя определённое время, используя концепцию, называемую политикой хранения. Также имеется возможность выполнять непрерывные запросы к оперативным данным для выполнения определённых операций. \cite{ryadi}

\paragraph{OpenTSDB}
OpenTSDB состоит из демона временных рядов, а также набора утилит командной строки. Взаимодействие с OpenTSDB в первую очередь достигается путем запуска одного или нескольких независимых демонов.

Демон использует базу данных с открытым исходным кодом HBase или службу Google Bigtable для хранения и получения данных временных рядов. Схема данных высоко оптимизирована для быстрого объединения аналогичных временных рядов, чтобы минимизировать пространство хранения. Пользователям никогда не требуется прямой доступ к базовому хранилищу. Можно общаться с демоном через протокол telnet, HTTP API или простой встроенный графический интерфейс.

\subsubsection{Реляционные базы данных}
Реляционная база данных --- это организованный по реляционной модели набор таблиц, в которых каждая ячейка этих таблиц имеет некоторое соответствующее описание. \cite{relationki}

Использование реляционной модели предполагает возможность идентификации элементов по совокупности уникальных идентификаторов: имя столбца, первичный ключ. Для построения логической связи между строками и ячейками разных таблиц используются внешние ключи. \cite{relationki}

Среди подобных СУБД, которые основаны на реляционных базах данных выделяются: Oracle, PostgreSQL.

Каждая из указанных СУБД имеет некоторые отличительные особенности. Так, например, PostgreSQL поддерживает вставки кода, написанного на языке программирования Python, в тело процедуры. Однако выделить среди данных особенностей важных для данной работы не представляется возможным.

\subsubsection*{Вывод}
Базы данных временных рядов являются наиболее подходящими для решаемой задачи, так как они нацелены на хранение, извлечение и анализ большого количества статистических данных, в которых имеются временные метки.

Для организации хранения данных будет использоваться СУБД InfluxDB, так как она является одной из самых популярных, среди известных баз данных временных рядов, а также по той причине, что поддержка данной СУБД всё ещё не прекращена на сегодняшний день.

\subsection{Данные для формирования нечеткой модели}

\subsection{Сведения о модулях}
Ахахахаха, да вы чо ржоте да))

\subsection{Структура и состав классов}
Внатуре не смешно уже хватит

\subsection{Развертывание системы}

