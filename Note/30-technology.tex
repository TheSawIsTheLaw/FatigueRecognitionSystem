\section{Технологическая часть}

\subsection{Используемые методы исследования корреляции}
Корреляционные методы позволяют решать задачи определения изменения зависимой переменной под влиянием одного или набора факторов, установки тесноты связи результативного признака с отдельным фактором, включенным в анализ. Также данные методы предоставляют возможность оценки общего объема вариации зависимой переменной и определения роли каждого фактора, а также проведения статистической оценки выборочных показателей корреляционной связи. \cite{correlInEco}

Характеристикой корреляционной зависимости является статистическая величина, называемая коэффициентом корреляции. \cite{corelMethod}

\subsubsection{Парная корреляция}

Для парной корреляции (фактор и отклик имеют нормальные распределения), коэффициент корреляции вычисляется по формуле \cite{corelMethod}:

\begin{equation}
\label{eq:corelPara}
\rho = \frac{M\left \{\left[x-M\left(X\right)\right]\left[y-M(Y)\right]\right \}}{\sqrt{M\left[x-M\left(X\right)^2\right]M\left[y-M\left(Y\right)^2\right]}} = \frac{K_{XY}}{\sqrt{D\left(X\right)D\left (Y\right)}},
\end{equation}
\eqexplSetIntro{где}
\begin{eqexpl}[15mm]
\item{$K_{XY}$} корреляционный момент, представляющий собой математическое ожидание произведения отклонений значений $x$ и $y$ случайных величин $X$ и $Y$ от их математических ожиданий $M(X)$ и $M(Y)$;
\item{$D(X)$} дисперсия случайной величины $X$;
\item{$D(Y)$} дисперсия случайной величины $Y$.
\end{eqexpl}

На практике случайные величины представляются ограниченным числом значений, поэтому вместо значения коэффициента корреляции $\rho$ используется его оценка $r$, которая рассчитывается с использованием выборочных характеристик отклика и фактора \cite{correlMethod}:

\begin{equation}
\label{eq:corelMark}
r = \frac{\sum_{i = 1}^{n}(x_i - \overline{x})(y_i - \overline{y}))}{(n - 1)\overline{S}_X \overline{S}_Y},
\end{equation}
\eqexplSetIntro{где}
\begin{eqexpl}[15mm]
\item{$\overline{x}$ и $\overline{y}$} средние выборочные значения фактора и отклика;
\item{$\overline{S}_X$ и $\overline{S}_Y$} выборочные стандартные отклонения отклика и фактора;
\item{$n$} число наблюдений.
\end{eqexpl}

Следует отметить, что значение $r$ лежит на интервале от $-1$ до $+1$. В случае, когда $r=\pm1$ зависимость между фактором и откликом является функциональной. Также положительное значение коэффициента корреляции указывает на возрастание отклика с увеличением фактора, когда отрицательные значение свидетельствует об убывании $Y$ при возрастании $X$. Равенство коэффициента парной корреляции нулю указывает, что взаимосвязь не является линейной. \cite{corelMethod}

\subsubsection{Множественная корреляция}
Множественная корреляция определяет обусловленность некоторого признака одновременным действием нескольких других признаков. Взаимодействия отклика с каждым из факторов и факторов между собой отображают в виде матрицы корреляции. \cite{corelMethod}

\begin{table}[H]
\begin{center}
\caption{\label{table: corelMatrix} Матрица корреляции}
\begin{tabular}{l||llllll}
      & $Y$         & $X_1$         & $...$ & $X_j$          & $...$ & $X_m$             \\ \hline\hline
$Y$   & $1$         & $r_{Y,X_1}$   & $...$ & $r_{Y,X_j}$    & $...$ & $r_{Y,X_m}$    \\
$X_1$ & $r_{Y,X_1}$ & $1 $          & $...$ & $r_{X_1,X_j}$  & $...$ & $r_{X_1,X_m}$ \\
$...$ & $...$       & $...$         & $1$   & $...$          & $...$ & $...$              \\
$X_j$ & $r_{Y,X_j}$ & $r_{X_1,X_j}$ & $...$ & $1$            & $...$ & $r_{X_m,X_m}$ \\
$...$ & $...$       & $...$         & $...$ & $...$          & $1$   & $...$              \\
$X_m$ & $r_{Y,X_m}$ & $r_{X_1,X_m}$ & $...$ & $r_{X_j,X_m}$  & $...$ & $1$               
\end{tabular}
\end{center}
\end{table}

Коэффициент множественной корреляции определяют из предположения, что отклик связан с факторами линейной зависимостью, причём коэффициент вычисляется по следующей формуле:
\begin{equation}
\label{eq:fuckMark}
R = \sqrt{1 - \frac{\Delta_{YX}}{\Delta_{XX}}},
\end{equation}
\eqexplSetIntro{где}
\begin{eqexpl}[15mm]
\item{$\Delta_{YX}$} определитель матрицы корреляции;
\item{$\Delta_{XX}$} определитель матрицы, получаемой из матрицы корреляции вычеркиванием первой строки и первого столбца.
\end{eqexpl}

Значимость множественного коэффициента корреляции проверяется с импользованием критерия Фишера:

\begin{equation}
\label{eq:fisherCrit}
F_\rho = \frac{R^2}{1-R^2} \frac{n - m - 2}{m} > F\left[{\alpha;m;n-m-2}\right] ,
\end{equation}
\eqexplSetIntro{где}
\begin{eqexpl}[15mm]
\item{$\Delta_{YX}$} определитель матрицы корреляции;
\item{$\Delta_{XX}$} определитель матрицы, получаемой из матрицы корреляции вычеркиванием первой строки и первого столбца.
\end{eqexpl}

\pagebreak