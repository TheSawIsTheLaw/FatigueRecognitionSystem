\section{Апробирование метода систематического распознавания усталости на автоматизированном рабочем месте}

\subsection{Технические характеристики}

Технические характеристики ЭВМ, на котором выполнялись исследования:

\begin{itemize}[leftmargin=1.6\parindent, after=\vspace{5pt}]
\item операционная система: Manjaro Linux (5.13.19-2);
\item оперативная память: 16 гигабайт;
\item процессор: Intel(R) Core(TM) i7-10510U CPU @ 1.80GHz.
\end{itemize}

\subsection{Сравнение количества успешных определений работоспособности пользователя путем варьирования фактора нечеткости}

\subsubsection{Предварительные условия}

В данном исследовании используется набор данных, полученных на студенте, выполняющим письменную работу. Во время сбора данных объекту исследования было разрешено отвлекаться на отдых, выполняя действия развлекательного характера в сети Интернет.

Полные данные, в силу их количества, приведены быть не могут. В листингах \ref{lst:testKeysLogging} (приложение В, с. \pageref{chp:application-c}) и \ref{lst:testMouseClicksLogging} (приложение В, с. \pageref{chp:application-c}) приведена часть полученных по пользователю данных.

В качестве варьируемого параметра принимается критерий нечеткости при проведении кластеризации методом c-средних.

По тесту на реакцию было определено, что передаваемые анализатору данные должны говорить о том, что пользователю требуется отдых.

\subsubsection[Сравнение количества успешных определений работоспособности по \newline клавиатуре]{Сравнение количества успешных определений работоспособности по клавиатуре}

В таблице \ref{table:time2} представлены результаты определений работоспособности пользователя по заданной выборке с использованием различных факторов не \- четкости. Для каждого значения фактора проводилось 10 определений состояния.

\begin{table}[H]
	\begin{center}
		\caption{\label{table:time2} Количество определений состояния системой в зависимости от значения критерия нечеткости.}
\begin{tabular}{|c|ccc|}
\hline
\multirow{2}{*}{Значение критерия нечеткости} & \multicolumn{3}{c|}{Количество определений состояния}\\ \cline{2-4} & \multicolumn{1}{c|}{Неопределен} & \multicolumn{1}{c|}{Устал} & Работоспособен \\ \hline
1.5& \multicolumn{1}{c|}{10}            & \multicolumn{1}{c|}{0}      &0 \\ \hline
2.0& \multicolumn{1}{c|}{10}            & \multicolumn{1}{c|}{0}     &0 \\ \hline
2.5& \multicolumn{1}{c|}{10}            & \multicolumn{1}{c|}{0}     &0 \\ \hline
3.0& \multicolumn{1}{c|}{10}            & \multicolumn{1}{c|}{0}      &0 \\ \hline
3.5& \multicolumn{1}{c|}{10}            & \multicolumn{1}{c|}{0}      &0 \\ \hline
4.0& \multicolumn{1}{c|}{10}            & \multicolumn{1}{c|}{0}      &0 \\ \hline
4.5& \multicolumn{1}{c|}{0}            & \multicolumn{1}{c|}{10}      &0 \\ \hline
5.0& \multicolumn{1}{c|}{0}            & \multicolumn{1}{c|}{10}      &0 \\ \hline
5.5& \multicolumn{1}{c|}{0}            & \multicolumn{1}{c|}{10}      &0 \\ \hline
6.0& \multicolumn{1}{c|}{0}            & \multicolumn{1}{c|}{10}      &0 \\ \hline
6.5& \multicolumn{1}{c|}{0}            & \multicolumn{1}{c|}{10}      &0 \\ \hline
7.0& \multicolumn{1}{c|}{0}            & \multicolumn{1}{c|}{10}      &0 \\ \hline
7.5& \multicolumn{1}{c|}{0}            & \multicolumn{1}{c|}{10}      &0 \\ \hline
8.0& \multicolumn{1}{c|}{0}            & \multicolumn{1}{c|}{10}      &0 \\ \hline
8.5& \multicolumn{1}{c|}{0}            & \multicolumn{1}{c|}{10}      &0 \\ \hline
9.0& \multicolumn{1}{c|}{0}            & \multicolumn{1}{c|}{10}      &0 \\ \hline
9.5& \multicolumn{1}{c|}{0}            & \multicolumn{1}{c|}{10}      &0 \\ \hline
10.0& \multicolumn{1}{c|}{0}            & \multicolumn{1}{c|}{10}     &0 \\ \hline
\end{tabular}
	\end{center}
\end{table}

Таким образом, можно сделать вывод о том, что наилучшими значениями, принимаемыми за критерий нечеткости, могут служить значения больше или равные 4.5, так как в таком случае шанс успешного определения состояния пользователя составляет $100\%$.

\subsubsection*{Вывод}

В результате проведенного исследования было определено, что на заданной выборке наиболее точными в использовании являются факторы нечеткости, лежащие на отрезке от 4.5 до 10.0, так как при их использовании имеем $100\%$ верных распознаваний.

\subsubsection[Сравнение количества успешных определений работоспособности по \newline мыши]{Сравнение количества успешных определений работоспособности по мыши}

В таблице \ref{table:time3} представлены результаты определений работоспособности пользователя по заданной выборке с использованием различных факторов не \- четкости. Для каждого значения фактора проводилось 10 определений состояния.

\begin{table}[H]
	\begin{center}
		\caption{\label{table:time3} Количество определений состояния системой в зависимости от значения критерия нечеткости.}
\begin{tabular}{|c|ccc|}
\hline
\multirow{2}{*}{Значение критерия нечеткости} & \multicolumn{3}{c|}{Количество определений состояния}\\ \cline{2-4} & \multicolumn{1}{c|}{Неопределен} & \multicolumn{1}{c|}{Устал} & Работоспособен \\ \hline
1.5& \multicolumn{1}{c|}{0}            & \multicolumn{1}{c|}{10}    &0 \\ \hline
2.0& \multicolumn{1}{c|}{0}            & \multicolumn{1}{c|}{10}      &0 \\ \hline
2.5& \multicolumn{1}{c|}{0}            & \multicolumn{1}{c|}{10}      &0 \\ \hline
3.0& \multicolumn{1}{c|}{0}            & \multicolumn{1}{c|}{10}      &0 \\ \hline
3.5& \multicolumn{1}{c|}{0}            & \multicolumn{1}{c|}{10}      &0 \\ \hline
4.0& \multicolumn{1}{c|}{10}            & \multicolumn{1}{c|}{0}      &0 \\ \hline
4.5& \multicolumn{1}{c|}{8}            & \multicolumn{1}{c|}{2}      &0 \\ \hline
5.0& \multicolumn{1}{c|}{6}            & \multicolumn{1}{c|}{4}      &0 \\ \hline
5.5& \multicolumn{1}{c|}{9}            & \multicolumn{1}{c|}{1}      &0 \\ \hline
6.0& \multicolumn{1}{c|}{9}            & \multicolumn{1}{c|}{1}      &0 \\ \hline
6.5& \multicolumn{1}{c|}{10}            & \multicolumn{1}{c|}{0}      &0 \\ \hline
7.0& \multicolumn{1}{c|}{10}            & \multicolumn{1}{c|}{0}      &0 \\ \hline
7.5& \multicolumn{1}{c|}{10}            & \multicolumn{1}{c|}{0}      &0 \\ \hline
8.0& \multicolumn{1}{c|}{0}            & \multicolumn{1}{c|}{10}      &0 \\ \hline
8.5& \multicolumn{1}{c|}{10}            & \multicolumn{1}{c|}{0}      &0 \\ \hline
9.0& \multicolumn{1}{c|}{0}            & \multicolumn{1}{c|}{10}      &0 \\ \hline
9.5& \multicolumn{1}{c|}{0}            & \multicolumn{1}{c|}{10}      &0 \\ \hline
10.0& \multicolumn{1}{c|}{0}            & \multicolumn{1}{c|}{10}     &0 \\ \hline
\end{tabular}
	\end{center}
\end{table}

Таким образом, можно сделать вывод о том, что наилучшими значениями, принимаемыми за критерий нечеткости, могут служить значения 1.5 -- 3.5, 8.0, 9.0 -- 10.0, так как в таком случае шанс успешного определения состояния пользователя составляет $100\%$.

\subsubsection*{Вывод}
В результате проведенного исследования было определено, что на заданной выборке наиболее точными в использовании являются факторы нечеткости 1.5 -- 3.5, 8.0, 9.0 -- 10.0. Иные факторы не позволяют с той же точностью в $100\%$ сказать о том, что в данный момент пользователю требуется отдых. Причем критерий нечеткости 5.0 позволил с шансом в $40\%$ распознать опасное состояние оператора, в то время как оставшиеся критерии имели шанс в $20\%$ и $10\%$.

\subsection*{Вывод}
Исследования показали, что при распознавании усталости с использованием клавиатуры при варьировании фактора нечеткости было получено $\approx 67\%$ верных результатов, что на $17\%$ больше, чем при использовании мыши с точностью $50\%$. Таким образом, на заданной выборке более эффективным устройством для распознавания усталости оказалась клавиатура. При этом для клавиатуры наиболее точными факторами нечеткости являются значения, лежащие на отрезке от 4.5 до 10.0, которые позволили с точностью в $100\%$ определить состояние оператора. Для мыши наиболее точными факторами нечеткости являются 1.5--3.5, 8.0, 9.0--10.0, которые также позволили с точностью в 100\% определить состояние оператора.

\pagebreak